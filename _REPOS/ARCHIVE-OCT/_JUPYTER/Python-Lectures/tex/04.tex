
% Default to the notebook output style

    


% Inherit from the specified cell style.




    
    
    
    \definecolor{orange}{cmyk}{0,0.4,0.8,0.2}
    \definecolor{darkorange}{rgb}{.71,0.21,0.01}
    \definecolor{darkgreen}{rgb}{.12,.54,.11}
    \definecolor{myteal}{rgb}{.26, .44, .56}
    \definecolor{gray}{gray}{0.45}
    \definecolor{lightgray}{gray}{.95}
    \definecolor{mediumgray}{gray}{.8}
    \definecolor{inputbackground}{rgb}{.95, .95, .85}
    \definecolor{outputbackground}{rgb}{.95, .95, .95}
    \definecolor{traceback}{rgb}{1, .95, .95}
    % ansi colors
    \definecolor{red}{rgb}{.6,0,0}
    \definecolor{green}{rgb}{0,.65,0}
    \definecolor{brown}{rgb}{0.6,0.6,0}
    \definecolor{blue}{rgb}{0,.145,.698}
    \definecolor{purple}{rgb}{.698,.145,.698}
    \definecolor{cyan}{rgb}{0,.698,.698}
    \definecolor{lightgray}{gray}{0.5}
    
    % bright ansi colors
    \definecolor{darkgray}{gray}{0.25}
    \definecolor{lightred}{rgb}{1.0,0.39,0.28}
    \definecolor{lightgreen}{rgb}{0.48,0.99,0.0}
    \definecolor{lightblue}{rgb}{0.53,0.81,0.92}
    \definecolor{lightpurple}{rgb}{0.87,0.63,0.87}
    \definecolor{lightcyan}{rgb}{0.5,1.0,0.83}
    
    % commands and environments needed by pandoc snippets
    % extracted from the output of `pandoc -s`
    \DefineVerbatimEnvironment{Highlighting}{Verbatim}{commandchars=\\\{\}}
    % Add ',fontsize=\small' for more characters per line
    \newenvironment{Shaded}{}{}
    \newcommand{\KeywordTok}[1]{\textcolor[rgb]{0.00,0.44,0.13}{\textbf{{#1}}}}
    \newcommand{\DataTypeTok}[1]{\textcolor[rgb]{0.56,0.13,0.00}{{#1}}}
    \newcommand{\DecValTok}[1]{\textcolor[rgb]{0.25,0.63,0.44}{{#1}}}
    \newcommand{\BaseNTok}[1]{\textcolor[rgb]{0.25,0.63,0.44}{{#1}}}
    \newcommand{\FloatTok}[1]{\textcolor[rgb]{0.25,0.63,0.44}{{#1}}}
    \newcommand{\CharTok}[1]{\textcolor[rgb]{0.25,0.44,0.63}{{#1}}}
    \newcommand{\StringTok}[1]{\textcolor[rgb]{0.25,0.44,0.63}{{#1}}}
    \newcommand{\CommentTok}[1]{\textcolor[rgb]{0.38,0.63,0.69}{\textit{{#1}}}}
    \newcommand{\OtherTok}[1]{\textcolor[rgb]{0.00,0.44,0.13}{{#1}}}
    \newcommand{\AlertTok}[1]{\textcolor[rgb]{1.00,0.00,0.00}{\textbf{{#1}}}}
    \newcommand{\FunctionTok}[1]{\textcolor[rgb]{0.02,0.16,0.49}{{#1}}}
    \newcommand{\RegionMarkerTok}[1]{{#1}}
    \newcommand{\ErrorTok}[1]{\textcolor[rgb]{1.00,0.00,0.00}{\textbf{{#1}}}}
    \newcommand{\NormalTok}[1]{{#1}}
    
    % Define a nice break command that doesn't care if a line doesn't already
    % exist.
    \def\br{\hspace*{\fill} \\* }
    % Math Jax compatability definitions
    \def\gt{>}
    \def\lt{<}
    % Document parameters
    \title{}
    
    
    

    % Pygments definitions
    
\makeatletter
\def\PY@reset{\let\PY@it=\relax \let\PY@bf=\relax%
    \let\PY@ul=\relax \let\PY@tc=\relax%
    \let\PY@bc=\relax \let\PY@ff=\relax}
\def\PY@tok#1{\csname PY@tok@#1\endcsname}
\def\PY@toks#1+{\ifx\relax#1\empty\else%
    \PY@tok{#1}\expandafter\PY@toks\fi}
\def\PY@do#1{\PY@bc{\PY@tc{\PY@ul{%
    \PY@it{\PY@bf{\PY@ff{#1}}}}}}}
\def\PY#1#2{\PY@reset\PY@toks#1+\relax+\PY@do{#2}}

\expandafter\def\csname PY@tok@gd\endcsname{\def\PY@tc##1{\textcolor[rgb]{0.63,0.00,0.00}{##1}}}
\expandafter\def\csname PY@tok@gu\endcsname{\let\PY@bf=\textbf\def\PY@tc##1{\textcolor[rgb]{0.50,0.00,0.50}{##1}}}
\expandafter\def\csname PY@tok@gt\endcsname{\def\PY@tc##1{\textcolor[rgb]{0.00,0.27,0.87}{##1}}}
\expandafter\def\csname PY@tok@gs\endcsname{\let\PY@bf=\textbf}
\expandafter\def\csname PY@tok@gr\endcsname{\def\PY@tc##1{\textcolor[rgb]{1.00,0.00,0.00}{##1}}}
\expandafter\def\csname PY@tok@cm\endcsname{\let\PY@it=\textit\def\PY@tc##1{\textcolor[rgb]{0.25,0.50,0.50}{##1}}}
\expandafter\def\csname PY@tok@vg\endcsname{\def\PY@tc##1{\textcolor[rgb]{0.10,0.09,0.49}{##1}}}
\expandafter\def\csname PY@tok@m\endcsname{\def\PY@tc##1{\textcolor[rgb]{0.40,0.40,0.40}{##1}}}
\expandafter\def\csname PY@tok@mh\endcsname{\def\PY@tc##1{\textcolor[rgb]{0.40,0.40,0.40}{##1}}}
\expandafter\def\csname PY@tok@go\endcsname{\def\PY@tc##1{\textcolor[rgb]{0.53,0.53,0.53}{##1}}}
\expandafter\def\csname PY@tok@ge\endcsname{\let\PY@it=\textit}
\expandafter\def\csname PY@tok@vc\endcsname{\def\PY@tc##1{\textcolor[rgb]{0.10,0.09,0.49}{##1}}}
\expandafter\def\csname PY@tok@il\endcsname{\def\PY@tc##1{\textcolor[rgb]{0.40,0.40,0.40}{##1}}}
\expandafter\def\csname PY@tok@cs\endcsname{\let\PY@it=\textit\def\PY@tc##1{\textcolor[rgb]{0.25,0.50,0.50}{##1}}}
\expandafter\def\csname PY@tok@cp\endcsname{\def\PY@tc##1{\textcolor[rgb]{0.74,0.48,0.00}{##1}}}
\expandafter\def\csname PY@tok@gi\endcsname{\def\PY@tc##1{\textcolor[rgb]{0.00,0.63,0.00}{##1}}}
\expandafter\def\csname PY@tok@gh\endcsname{\let\PY@bf=\textbf\def\PY@tc##1{\textcolor[rgb]{0.00,0.00,0.50}{##1}}}
\expandafter\def\csname PY@tok@ni\endcsname{\let\PY@bf=\textbf\def\PY@tc##1{\textcolor[rgb]{0.60,0.60,0.60}{##1}}}
\expandafter\def\csname PY@tok@nl\endcsname{\def\PY@tc##1{\textcolor[rgb]{0.63,0.63,0.00}{##1}}}
\expandafter\def\csname PY@tok@nn\endcsname{\let\PY@bf=\textbf\def\PY@tc##1{\textcolor[rgb]{0.00,0.00,1.00}{##1}}}
\expandafter\def\csname PY@tok@no\endcsname{\def\PY@tc##1{\textcolor[rgb]{0.53,0.00,0.00}{##1}}}
\expandafter\def\csname PY@tok@na\endcsname{\def\PY@tc##1{\textcolor[rgb]{0.49,0.56,0.16}{##1}}}
\expandafter\def\csname PY@tok@nb\endcsname{\def\PY@tc##1{\textcolor[rgb]{0.00,0.50,0.00}{##1}}}
\expandafter\def\csname PY@tok@nc\endcsname{\let\PY@bf=\textbf\def\PY@tc##1{\textcolor[rgb]{0.00,0.00,1.00}{##1}}}
\expandafter\def\csname PY@tok@nd\endcsname{\def\PY@tc##1{\textcolor[rgb]{0.67,0.13,1.00}{##1}}}
\expandafter\def\csname PY@tok@ne\endcsname{\let\PY@bf=\textbf\def\PY@tc##1{\textcolor[rgb]{0.82,0.25,0.23}{##1}}}
\expandafter\def\csname PY@tok@nf\endcsname{\def\PY@tc##1{\textcolor[rgb]{0.00,0.00,1.00}{##1}}}
\expandafter\def\csname PY@tok@si\endcsname{\let\PY@bf=\textbf\def\PY@tc##1{\textcolor[rgb]{0.73,0.40,0.53}{##1}}}
\expandafter\def\csname PY@tok@s2\endcsname{\def\PY@tc##1{\textcolor[rgb]{0.73,0.13,0.13}{##1}}}
\expandafter\def\csname PY@tok@vi\endcsname{\def\PY@tc##1{\textcolor[rgb]{0.10,0.09,0.49}{##1}}}
\expandafter\def\csname PY@tok@nt\endcsname{\let\PY@bf=\textbf\def\PY@tc##1{\textcolor[rgb]{0.00,0.50,0.00}{##1}}}
\expandafter\def\csname PY@tok@nv\endcsname{\def\PY@tc##1{\textcolor[rgb]{0.10,0.09,0.49}{##1}}}
\expandafter\def\csname PY@tok@s1\endcsname{\def\PY@tc##1{\textcolor[rgb]{0.73,0.13,0.13}{##1}}}
\expandafter\def\csname PY@tok@kd\endcsname{\let\PY@bf=\textbf\def\PY@tc##1{\textcolor[rgb]{0.00,0.50,0.00}{##1}}}
\expandafter\def\csname PY@tok@sh\endcsname{\def\PY@tc##1{\textcolor[rgb]{0.73,0.13,0.13}{##1}}}
\expandafter\def\csname PY@tok@sc\endcsname{\def\PY@tc##1{\textcolor[rgb]{0.73,0.13,0.13}{##1}}}
\expandafter\def\csname PY@tok@sx\endcsname{\def\PY@tc##1{\textcolor[rgb]{0.00,0.50,0.00}{##1}}}
\expandafter\def\csname PY@tok@bp\endcsname{\def\PY@tc##1{\textcolor[rgb]{0.00,0.50,0.00}{##1}}}
\expandafter\def\csname PY@tok@c1\endcsname{\let\PY@it=\textit\def\PY@tc##1{\textcolor[rgb]{0.25,0.50,0.50}{##1}}}
\expandafter\def\csname PY@tok@kc\endcsname{\let\PY@bf=\textbf\def\PY@tc##1{\textcolor[rgb]{0.00,0.50,0.00}{##1}}}
\expandafter\def\csname PY@tok@c\endcsname{\let\PY@it=\textit\def\PY@tc##1{\textcolor[rgb]{0.25,0.50,0.50}{##1}}}
\expandafter\def\csname PY@tok@mf\endcsname{\def\PY@tc##1{\textcolor[rgb]{0.40,0.40,0.40}{##1}}}
\expandafter\def\csname PY@tok@err\endcsname{\def\PY@bc##1{\setlength{\fboxsep}{0pt}\fcolorbox[rgb]{1.00,0.00,0.00}{1,1,1}{\strut ##1}}}
\expandafter\def\csname PY@tok@mb\endcsname{\def\PY@tc##1{\textcolor[rgb]{0.40,0.40,0.40}{##1}}}
\expandafter\def\csname PY@tok@ss\endcsname{\def\PY@tc##1{\textcolor[rgb]{0.10,0.09,0.49}{##1}}}
\expandafter\def\csname PY@tok@sr\endcsname{\def\PY@tc##1{\textcolor[rgb]{0.73,0.40,0.53}{##1}}}
\expandafter\def\csname PY@tok@mo\endcsname{\def\PY@tc##1{\textcolor[rgb]{0.40,0.40,0.40}{##1}}}
\expandafter\def\csname PY@tok@kn\endcsname{\let\PY@bf=\textbf\def\PY@tc##1{\textcolor[rgb]{0.00,0.50,0.00}{##1}}}
\expandafter\def\csname PY@tok@mi\endcsname{\def\PY@tc##1{\textcolor[rgb]{0.40,0.40,0.40}{##1}}}
\expandafter\def\csname PY@tok@gp\endcsname{\let\PY@bf=\textbf\def\PY@tc##1{\textcolor[rgb]{0.00,0.00,0.50}{##1}}}
\expandafter\def\csname PY@tok@o\endcsname{\def\PY@tc##1{\textcolor[rgb]{0.40,0.40,0.40}{##1}}}
\expandafter\def\csname PY@tok@kr\endcsname{\let\PY@bf=\textbf\def\PY@tc##1{\textcolor[rgb]{0.00,0.50,0.00}{##1}}}
\expandafter\def\csname PY@tok@s\endcsname{\def\PY@tc##1{\textcolor[rgb]{0.73,0.13,0.13}{##1}}}
\expandafter\def\csname PY@tok@kp\endcsname{\def\PY@tc##1{\textcolor[rgb]{0.00,0.50,0.00}{##1}}}
\expandafter\def\csname PY@tok@w\endcsname{\def\PY@tc##1{\textcolor[rgb]{0.73,0.73,0.73}{##1}}}
\expandafter\def\csname PY@tok@kt\endcsname{\def\PY@tc##1{\textcolor[rgb]{0.69,0.00,0.25}{##1}}}
\expandafter\def\csname PY@tok@ow\endcsname{\let\PY@bf=\textbf\def\PY@tc##1{\textcolor[rgb]{0.67,0.13,1.00}{##1}}}
\expandafter\def\csname PY@tok@sb\endcsname{\def\PY@tc##1{\textcolor[rgb]{0.73,0.13,0.13}{##1}}}
\expandafter\def\csname PY@tok@k\endcsname{\let\PY@bf=\textbf\def\PY@tc##1{\textcolor[rgb]{0.00,0.50,0.00}{##1}}}
\expandafter\def\csname PY@tok@se\endcsname{\let\PY@bf=\textbf\def\PY@tc##1{\textcolor[rgb]{0.73,0.40,0.13}{##1}}}
\expandafter\def\csname PY@tok@sd\endcsname{\let\PY@it=\textit\def\PY@tc##1{\textcolor[rgb]{0.73,0.13,0.13}{##1}}}

\def\PYZbs{\char`\\}
\def\PYZus{\char`\_}
\def\PYZob{\char`\{}
\def\PYZcb{\char`\}}
\def\PYZca{\char`\^}
\def\PYZam{\char`\&}
\def\PYZlt{\char`\<}
\def\PYZgt{\char`\>}
\def\PYZsh{\char`\#}
\def\PYZpc{\char`\%}
\def\PYZdl{\char`\$}
\def\PYZhy{\char`\-}
\def\PYZsq{\char`\'}
\def\PYZdq{\char`\"}
\def\PYZti{\char`\~}
% for compatibility with earlier versions
\def\PYZat{@}
\def\PYZlb{[}
\def\PYZrb{]}
\makeatother


    % Exact colors from NB
    \definecolor{incolor}{rgb}{0.0, 0.0, 0.5}
    \definecolor{outcolor}{rgb}{0.545, 0.0, 0.0}



    
    % Prevent overflowing lines due to hard-to-break entities
    \sloppy 
    % Setup hyperref package
    \hypersetup{
      breaklinks=true,  % so long urls are correctly broken across lines
      colorlinks=true,
      urlcolor=blue,
      linkcolor=darkorange,
      citecolor=darkgreen,
      }
    % Slightly bigger margins than the latex defaults
    
     

    \begin{document}
    
    
    \maketitle
    
    

    
    \subsection{Strings}\label{strings}

    Strings are ordered text based data which are represented by enclosing
the same in single/double/triple quotes.

    \begin{Verbatim}[commandchars=\\\{\}]
{\color{incolor}In [{\color{incolor}1}]:} \PY{n}{String0} \PY{o}{=} \PY{l+s}{\PYZsq{}}\PY{l+s}{Taj Mahal is beautiful}\PY{l+s}{\PYZsq{}}
        \PY{n}{String1} \PY{o}{=} \PY{l+s}{\PYZdq{}}\PY{l+s}{Taj Mahal is beautiful}\PY{l+s}{\PYZdq{}}
        \PY{n}{String2} \PY{o}{=} \PY{l+s}{\PYZsq{}\PYZsq{}\PYZsq{}}\PY{l+s}{Taj Mahal}
        \PY{l+s}{is}
        \PY{l+s}{beautiful}\PY{l+s}{\PYZsq{}\PYZsq{}\PYZsq{}}
\end{Verbatim}

    \begin{Verbatim}[commandchars=\\\{\}]
{\color{incolor}In [{\color{incolor}2}]:} \PY{k}{print} \PY{n}{String0} \PY{p}{,} \PY{n+nb}{type}\PY{p}{(}\PY{n}{String0}\PY{p}{)}
        \PY{k}{print} \PY{n}{String1}\PY{p}{,} \PY{n+nb}{type}\PY{p}{(}\PY{n}{String1}\PY{p}{)}
        \PY{k}{print} \PY{n}{String2}\PY{p}{,} \PY{n+nb}{type}\PY{p}{(}\PY{n}{String2}\PY{p}{)}
\end{Verbatim}

    \begin{Verbatim}[commandchars=\\\{\}]
Taj Mahal is beautiful <type 'str'>
Taj Mahal is beautiful <type 'str'>
Taj Mahal
is
beautiful <type 'str'>
    \end{Verbatim}

    String Indexing and Slicing are similar to Lists which was explained in
detail earlier.

    \begin{Verbatim}[commandchars=\\\{\}]
{\color{incolor}In [{\color{incolor}3}]:} \PY{k}{print} \PY{n}{String0}\PY{p}{[}\PY{l+m+mi}{4}\PY{p}{]}
        \PY{k}{print} \PY{n}{String0}\PY{p}{[}\PY{l+m+mi}{4}\PY{p}{:}\PY{p}{]}
\end{Verbatim}

    \begin{Verbatim}[commandchars=\\\{\}]
M
Mahal is beautiful
    \end{Verbatim}

    \subsubsection{Built-in Functions}\label{built-in-functions}

    \textbf{find( )} function returns the index value of the given data that
is to found in the string. If it is not found it returns \textbf{-1}.
Remember to not confuse the returned -1 for reverse indexing value.

    \begin{Verbatim}[commandchars=\\\{\}]
{\color{incolor}In [{\color{incolor}4}]:} \PY{k}{print} \PY{n}{String0}\PY{o}{.}\PY{n}{find}\PY{p}{(}\PY{l+s}{\PYZsq{}}\PY{l+s}{al}\PY{l+s}{\PYZsq{}}\PY{p}{)}
        \PY{k}{print} \PY{n}{String0}\PY{o}{.}\PY{n}{find}\PY{p}{(}\PY{l+s}{\PYZsq{}}\PY{l+s}{am}\PY{l+s}{\PYZsq{}}\PY{p}{)}
\end{Verbatim}

    \begin{Verbatim}[commandchars=\\\{\}]
7
-1
    \end{Verbatim}

    The index value returned is the index of the first element in the input
data.

    \begin{Verbatim}[commandchars=\\\{\}]
{\color{incolor}In [{\color{incolor}5}]:} \PY{k}{print} \PY{n}{String0}\PY{p}{[}\PY{l+m+mi}{7}\PY{p}{]}
\end{Verbatim}

    \begin{Verbatim}[commandchars=\\\{\}]
a
    \end{Verbatim}

    One can also input \textbf{find( )} function between which index values
it has to search.

    \begin{Verbatim}[commandchars=\\\{\}]
{\color{incolor}In [{\color{incolor}6}]:} \PY{k}{print} \PY{n}{String0}\PY{o}{.}\PY{n}{find}\PY{p}{(}\PY{l+s}{\PYZsq{}}\PY{l+s}{j}\PY{l+s}{\PYZsq{}}\PY{p}{,}\PY{l+m+mi}{1}\PY{p}{)}
        \PY{k}{print} \PY{n}{String0}\PY{o}{.}\PY{n}{find}\PY{p}{(}\PY{l+s}{\PYZsq{}}\PY{l+s}{j}\PY{l+s}{\PYZsq{}}\PY{p}{,}\PY{l+m+mi}{1}\PY{p}{,}\PY{l+m+mi}{3}\PY{p}{)}
\end{Verbatim}

    \begin{Verbatim}[commandchars=\\\{\}]
2
2
    \end{Verbatim}

    \textbf{capitalize( )} is used to capitalize the first element in the
string.

    \begin{Verbatim}[commandchars=\\\{\}]
{\color{incolor}In [{\color{incolor}7}]:} \PY{n}{String3} \PY{o}{=} \PY{l+s}{\PYZsq{}}\PY{l+s}{observe the first letter in this sentence.}\PY{l+s}{\PYZsq{}}
        \PY{k}{print} \PY{n}{String3}\PY{o}{.}\PY{n}{capitalize}\PY{p}{(}\PY{p}{)}
\end{Verbatim}

    \begin{Verbatim}[commandchars=\\\{\}]
Observe the first letter in this sentence.
    \end{Verbatim}

    \textbf{center( )} is used to center align the string by specifying the
field width.

    \begin{Verbatim}[commandchars=\\\{\}]
{\color{incolor}In [{\color{incolor}8}]:} \PY{n}{String0}\PY{o}{.}\PY{n}{center}\PY{p}{(}\PY{l+m+mi}{70}\PY{p}{)}
\end{Verbatim}

            \begin{Verbatim}[commandchars=\\\{\}]
{\color{outcolor}Out[{\color{outcolor}8}]:} '                        Taj Mahal is beautiful                        '
\end{Verbatim}
        
    One can also fill the left out spaces with any other character.

    \begin{Verbatim}[commandchars=\\\{\}]
{\color{incolor}In [{\color{incolor}9}]:} \PY{n}{String0}\PY{o}{.}\PY{n}{center}\PY{p}{(}\PY{l+m+mi}{70}\PY{p}{,}\PY{l+s}{\PYZsq{}}\PY{l+s}{\PYZhy{}}\PY{l+s}{\PYZsq{}}\PY{p}{)}
\end{Verbatim}

            \begin{Verbatim}[commandchars=\\\{\}]
{\color{outcolor}Out[{\color{outcolor}9}]:} '------------------------Taj Mahal is beautiful------------------------'
\end{Verbatim}
        
    \textbf{zfill( )} is used for zero padding by specifying the field
width.

    \begin{Verbatim}[commandchars=\\\{\}]
{\color{incolor}In [{\color{incolor}10}]:} \PY{n}{String0}\PY{o}{.}\PY{n}{zfill}\PY{p}{(}\PY{l+m+mi}{30}\PY{p}{)}
\end{Verbatim}

            \begin{Verbatim}[commandchars=\\\{\}]
{\color{outcolor}Out[{\color{outcolor}10}]:} '00000000Taj Mahal is beautiful'
\end{Verbatim}
        
    \textbf{expandtabs( )} allows you to change the spacing of the tab
character. `\t' which is by default set to 8 spaces.

    \begin{Verbatim}[commandchars=\\\{\}]
{\color{incolor}In [{\color{incolor}11}]:} \PY{n}{s} \PY{o}{=} \PY{l+s}{\PYZsq{}}\PY{l+s}{h}\PY{l+s+se}{\PYZbs{}t}\PY{l+s}{e}\PY{l+s+se}{\PYZbs{}t}\PY{l+s}{l}\PY{l+s+se}{\PYZbs{}t}\PY{l+s}{l}\PY{l+s+se}{\PYZbs{}t}\PY{l+s}{o}\PY{l+s}{\PYZsq{}}
         \PY{k}{print} \PY{n}{s}
         \PY{k}{print} \PY{n}{s}\PY{o}{.}\PY{n}{expandtabs}\PY{p}{(}\PY{l+m+mi}{1}\PY{p}{)}
         \PY{k}{print} \PY{n}{s}\PY{o}{.}\PY{n}{expandtabs}\PY{p}{(}\PY{p}{)}
\end{Verbatim}

    \begin{Verbatim}[commandchars=\\\{\}]
h	e	l	l	o
h e l l o
h       e       l       l       o
    \end{Verbatim}

    \textbf{index( )} works the same way as \textbf{find( )} function the
only difference is find returns `-1' when the input element is not found
in the string but \textbf{index( )} function throws a ValueError

    \begin{Verbatim}[commandchars=\\\{\}]
{\color{incolor}In [{\color{incolor}12}]:} \PY{k}{print} \PY{n}{String0}\PY{o}{.}\PY{n}{index}\PY{p}{(}\PY{l+s}{\PYZsq{}}\PY{l+s}{Taj}\PY{l+s}{\PYZsq{}}\PY{p}{)}
         \PY{k}{print} \PY{n}{String0}\PY{o}{.}\PY{n}{index}\PY{p}{(}\PY{l+s}{\PYZsq{}}\PY{l+s}{Mahal}\PY{l+s}{\PYZsq{}}\PY{p}{,}\PY{l+m+mi}{0}\PY{p}{)}
         \PY{k}{print} \PY{n}{String0}\PY{o}{.}\PY{n}{index}\PY{p}{(}\PY{l+s}{\PYZsq{}}\PY{l+s}{Mahal}\PY{l+s}{\PYZsq{}}\PY{p}{,}\PY{l+m+mi}{10}\PY{p}{,}\PY{l+m+mi}{20}\PY{p}{)}
\end{Verbatim}

    \begin{Verbatim}[commandchars=\\\{\}]
0
4
    \end{Verbatim}

    \begin{Verbatim}[commandchars=\\\{\}]

        ---------------------------------------------------------------------------

        ValueError                                Traceback (most recent call last)

        <ipython-input-12-6062e7a32deb> in <module>()
          1 print String0.index('Taj')
          2 print String0.index('Mahal',0)
    ----> 3 print String0.index('Mahal',10,20)
    

        ValueError: substring not found

    \end{Verbatim}

    \textbf{endswith( )} function is used to check if the given string ends
with the particular char which is given as input.

    \begin{Verbatim}[commandchars=\\\{\}]
{\color{incolor}In [{\color{incolor}13}]:} \PY{k}{print} \PY{n}{String0}\PY{o}{.}\PY{n}{endswith}\PY{p}{(}\PY{l+s}{\PYZsq{}}\PY{l+s}{y}\PY{l+s}{\PYZsq{}}\PY{p}{)}
\end{Verbatim}

    \begin{Verbatim}[commandchars=\\\{\}]
False
    \end{Verbatim}

    The start and stop index values can also be specified.

    \begin{Verbatim}[commandchars=\\\{\}]
{\color{incolor}In [{\color{incolor}14}]:} \PY{k}{print} \PY{n}{String0}\PY{o}{.}\PY{n}{endswith}\PY{p}{(}\PY{l+s}{\PYZsq{}}\PY{l+s}{l}\PY{l+s}{\PYZsq{}}\PY{p}{,}\PY{l+m+mi}{0}\PY{p}{)}
         \PY{k}{print} \PY{n}{String0}\PY{o}{.}\PY{n}{endswith}\PY{p}{(}\PY{l+s}{\PYZsq{}}\PY{l+s}{M}\PY{l+s}{\PYZsq{}}\PY{p}{,}\PY{l+m+mi}{0}\PY{p}{,}\PY{l+m+mi}{5}\PY{p}{)}
\end{Verbatim}

    \begin{Verbatim}[commandchars=\\\{\}]
True
True
    \end{Verbatim}

    \textbf{count( )} function counts the number of char in the given
string. The start and the stop index can also be specified or left
blank. (These are Implicit arguments which will be dealt in functions)

    \begin{Verbatim}[commandchars=\\\{\}]
{\color{incolor}In [{\color{incolor}15}]:} \PY{k}{print} \PY{n}{String0}\PY{o}{.}\PY{n}{count}\PY{p}{(}\PY{l+s}{\PYZsq{}}\PY{l+s}{a}\PY{l+s}{\PYZsq{}}\PY{p}{,}\PY{l+m+mi}{0}\PY{p}{)}
         \PY{k}{print} \PY{n}{String0}\PY{o}{.}\PY{n}{count}\PY{p}{(}\PY{l+s}{\PYZsq{}}\PY{l+s}{a}\PY{l+s}{\PYZsq{}}\PY{p}{,}\PY{l+m+mi}{5}\PY{p}{,}\PY{l+m+mi}{10}\PY{p}{)}
\end{Verbatim}

    \begin{Verbatim}[commandchars=\\\{\}]
4
2
    \end{Verbatim}

    \textbf{join( )} function is used add a char in between the elements of
the input string.

    \begin{Verbatim}[commandchars=\\\{\}]
{\color{incolor}In [{\color{incolor}16}]:} \PY{l+s}{\PYZsq{}}\PY{l+s}{a}\PY{l+s}{\PYZsq{}}\PY{o}{.}\PY{n}{join}\PY{p}{(}\PY{l+s}{\PYZsq{}}\PY{l+s}{*\PYZus{}\PYZhy{}}\PY{l+s}{\PYZsq{}}\PY{p}{)}
\end{Verbatim}

            \begin{Verbatim}[commandchars=\\\{\}]
{\color{outcolor}Out[{\color{outcolor}16}]:} '*a\_a-'
\end{Verbatim}
        
    '*\_-`is the input string and char 'a' is added in between each element

    \textbf{join( )} function can also be used to convert a list into a
string.

    \begin{Verbatim}[commandchars=\\\{\}]
{\color{incolor}In [{\color{incolor}17}]:} \PY{n}{a} \PY{o}{=} \PY{n+nb}{list}\PY{p}{(}\PY{n}{String0}\PY{p}{)}
         \PY{k}{print} \PY{n}{a}
         \PY{n}{b} \PY{o}{=} \PY{l+s}{\PYZsq{}}\PY{l+s}{\PYZsq{}}\PY{o}{.}\PY{n}{join}\PY{p}{(}\PY{n}{a}\PY{p}{)}
         \PY{k}{print} \PY{n}{b}
\end{Verbatim}

    \begin{Verbatim}[commandchars=\\\{\}]
['T', 'a', 'j', ' ', 'M', 'a', 'h', 'a', 'l', ' ', 'i', 's', ' ', 'b', 'e', 'a', 'u', 't', 'i', 'f', 'u', 'l']
Taj Mahal is beautiful
    \end{Verbatim}

    Before converting it into a string \textbf{join( )} function can be used
to insert any char in between the list elements.

    \begin{Verbatim}[commandchars=\\\{\}]
{\color{incolor}In [{\color{incolor}18}]:} \PY{n}{c} \PY{o}{=} \PY{l+s}{\PYZsq{}}\PY{l+s}{/}\PY{l+s}{\PYZsq{}}\PY{o}{.}\PY{n}{join}\PY{p}{(}\PY{n}{a}\PY{p}{)}\PY{p}{[}\PY{l+m+mi}{18}\PY{p}{:}\PY{p}{]}
         \PY{k}{print} \PY{n}{c}
\end{Verbatim}

    \begin{Verbatim}[commandchars=\\\{\}]
/i/s/ /b/e/a/u/t/i/f/u/l
    \end{Verbatim}

    \textbf{split( )} function is used to convert a string back to a list.
Think of it as the opposite of the \textbf{join()} function.

    \begin{Verbatim}[commandchars=\\\{\}]
{\color{incolor}In [{\color{incolor}19}]:} \PY{n}{d} \PY{o}{=} \PY{n}{c}\PY{o}{.}\PY{n}{split}\PY{p}{(}\PY{l+s}{\PYZsq{}}\PY{l+s}{/}\PY{l+s}{\PYZsq{}}\PY{p}{)}
         \PY{k}{print} \PY{n}{d}
\end{Verbatim}

    \begin{Verbatim}[commandchars=\\\{\}]
[' ', 'i', 's', ' ', 'b', 'e', 'a', 'u', 't', 'i', 'f', 'u', 'l']
    \end{Verbatim}

    In \textbf{split( )} function one can also specify the number of times
you want to split the string or the number of elements the new returned
list should conatin. The number of elements is always one more than the
specified number this is because it is split the number of times
specified.

    \begin{Verbatim}[commandchars=\\\{\}]
{\color{incolor}In [{\color{incolor}20}]:} \PY{n}{e} \PY{o}{=} \PY{n}{c}\PY{o}{.}\PY{n}{split}\PY{p}{(}\PY{l+s}{\PYZsq{}}\PY{l+s}{/}\PY{l+s}{\PYZsq{}}\PY{p}{,}\PY{l+m+mi}{3}\PY{p}{)}
         \PY{k}{print} \PY{n}{e}
         \PY{k}{print} \PY{n+nb}{len}\PY{p}{(}\PY{n}{e}\PY{p}{)}
\end{Verbatim}

    \begin{Verbatim}[commandchars=\\\{\}]
[' ', 'i', 's', ' /b/e/a/u/t/i/f/u/l']
4
    \end{Verbatim}

    \textbf{lower( )} converts any capital letter to small letter.

    \begin{Verbatim}[commandchars=\\\{\}]
{\color{incolor}In [{\color{incolor}21}]:} \PY{k}{print} \PY{n}{String0}
         \PY{k}{print} \PY{n}{String0}\PY{o}{.}\PY{n}{lower}\PY{p}{(}\PY{p}{)}
\end{Verbatim}

    \begin{Verbatim}[commandchars=\\\{\}]
Taj Mahal is beautiful
taj mahal is beautiful
    \end{Verbatim}

    \textbf{upper( )} converts any small letter to capital letter.

    \begin{Verbatim}[commandchars=\\\{\}]
{\color{incolor}In [{\color{incolor}22}]:} \PY{n}{String0}\PY{o}{.}\PY{n}{upper}\PY{p}{(}\PY{p}{)}
\end{Verbatim}

            \begin{Verbatim}[commandchars=\\\{\}]
{\color{outcolor}Out[{\color{outcolor}22}]:} 'TAJ MAHAL IS BEAUTIFUL'
\end{Verbatim}
        
    \textbf{replace( )} function replaces the element with another element.

    \begin{Verbatim}[commandchars=\\\{\}]
{\color{incolor}In [{\color{incolor}23}]:} \PY{n}{String0}\PY{o}{.}\PY{n}{replace}\PY{p}{(}\PY{l+s}{\PYZsq{}}\PY{l+s}{Taj Mahal}\PY{l+s}{\PYZsq{}}\PY{p}{,}\PY{l+s}{\PYZsq{}}\PY{l+s}{Bengaluru}\PY{l+s}{\PYZsq{}}\PY{p}{)}
\end{Verbatim}

            \begin{Verbatim}[commandchars=\\\{\}]
{\color{outcolor}Out[{\color{outcolor}23}]:} 'Bengaluru is beautiful'
\end{Verbatim}
        
    \textbf{strip( )} function is used to delete elements from the right end
and the left end which is not required.

    \begin{Verbatim}[commandchars=\\\{\}]
{\color{incolor}In [{\color{incolor}24}]:} \PY{n}{f} \PY{o}{=} \PY{l+s}{\PYZsq{}}\PY{l+s}{    hello      }\PY{l+s}{\PYZsq{}}
\end{Verbatim}

    If no char is specified then it will delete all the spaces that is
present in the right and left hand side of the data.

    \begin{Verbatim}[commandchars=\\\{\}]
{\color{incolor}In [{\color{incolor}25}]:} \PY{n}{f}\PY{o}{.}\PY{n}{strip}\PY{p}{(}\PY{p}{)}
\end{Verbatim}

            \begin{Verbatim}[commandchars=\\\{\}]
{\color{outcolor}Out[{\color{outcolor}25}]:} 'hello'
\end{Verbatim}
        
    \textbf{strip( )} function, when a char is specified then it deletes
that char if it is present in the two ends of the specified string.

    \begin{Verbatim}[commandchars=\\\{\}]
{\color{incolor}In [{\color{incolor}26}]:} \PY{n}{f} \PY{o}{=} \PY{l+s}{\PYZsq{}}\PY{l+s}{   ***\PYZhy{}\PYZhy{}\PYZhy{}\PYZhy{}hello\PYZhy{}\PYZhy{}\PYZhy{}*******     }\PY{l+s}{\PYZsq{}}
\end{Verbatim}

    \begin{Verbatim}[commandchars=\\\{\}]
{\color{incolor}In [{\color{incolor}27}]:} \PY{n}{f}\PY{o}{.}\PY{n}{strip}\PY{p}{(}\PY{l+s}{\PYZsq{}}\PY{l+s}{*}\PY{l+s}{\PYZsq{}}\PY{p}{)}
\end{Verbatim}

            \begin{Verbatim}[commandchars=\\\{\}]
{\color{outcolor}Out[{\color{outcolor}27}]:} '   ***----hello---*******     '
\end{Verbatim}
        
    The asterisk had to be deleted but is not. This is because there is a
space in both the right and left hand side. So in strip function. The
characters need to be inputted in the specific order in which they are
present.

    \begin{Verbatim}[commandchars=\\\{\}]
{\color{incolor}In [{\color{incolor}28}]:} \PY{k}{print} \PY{n}{f}\PY{o}{.}\PY{n}{strip}\PY{p}{(}\PY{l+s}{\PYZsq{}}\PY{l+s}{ *}\PY{l+s}{\PYZsq{}}\PY{p}{)}
         \PY{k}{print} \PY{n}{f}\PY{o}{.}\PY{n}{strip}\PY{p}{(}\PY{l+s}{\PYZsq{}}\PY{l+s}{ *\PYZhy{}}\PY{l+s}{\PYZsq{}}\PY{p}{)}
\end{Verbatim}

    \begin{Verbatim}[commandchars=\\\{\}]
----hello---
hello
    \end{Verbatim}

    \textbf{lstrip( )} and \textbf{rstrip( )} function have the same
functionality as strip function but the only difference is
\textbf{lstrip( )} deletes only towards the left side and
\textbf{rstrip( )} towards the right.

    \begin{Verbatim}[commandchars=\\\{\}]
{\color{incolor}In [{\color{incolor}29}]:} \PY{k}{print} \PY{n}{f}\PY{o}{.}\PY{n}{lstrip}\PY{p}{(}\PY{l+s}{\PYZsq{}}\PY{l+s}{ *}\PY{l+s}{\PYZsq{}}\PY{p}{)}
         \PY{k}{print} \PY{n}{f}\PY{o}{.}\PY{n}{rstrip}\PY{p}{(}\PY{l+s}{\PYZsq{}}\PY{l+s}{ *}\PY{l+s}{\PYZsq{}}\PY{p}{)}
\end{Verbatim}

    \begin{Verbatim}[commandchars=\\\{\}]
----hello---*******     
   ***----hello---
    \end{Verbatim}

    \subsection{Dictionaries}\label{dictionaries}

    Dictionaries are more used like a database because here you can index a
particular sequence with your user defined string.

    To define a dictionary, equate a variable to \{ \} or dict()

    \begin{Verbatim}[commandchars=\\\{\}]
{\color{incolor}In [{\color{incolor}30}]:} \PY{n}{d0} \PY{o}{=} \PY{p}{\PYZob{}}\PY{p}{\PYZcb{}}
         \PY{n}{d1} \PY{o}{=} \PY{n+nb}{dict}\PY{p}{(}\PY{p}{)}
         \PY{k}{print} \PY{n+nb}{type}\PY{p}{(}\PY{n}{d0}\PY{p}{)}\PY{p}{,} \PY{n+nb}{type}\PY{p}{(}\PY{n}{d1}\PY{p}{)}
\end{Verbatim}

    \begin{Verbatim}[commandchars=\\\{\}]
<type 'dict'> <type 'dict'>
    \end{Verbatim}

    Dictionary works somewhat like a list but with an added capability of
assigning it's own index style.

    \begin{Verbatim}[commandchars=\\\{\}]
{\color{incolor}In [{\color{incolor}31}]:} \PY{n}{d0}\PY{p}{[}\PY{l+s}{\PYZsq{}}\PY{l+s}{One}\PY{l+s}{\PYZsq{}}\PY{p}{]} \PY{o}{=} \PY{l+m+mi}{1}
         \PY{n}{d0}\PY{p}{[}\PY{l+s}{\PYZsq{}}\PY{l+s}{OneTwo}\PY{l+s}{\PYZsq{}}\PY{p}{]} \PY{o}{=} \PY{l+m+mi}{12} 
         \PY{k}{print} \PY{n}{d0}
\end{Verbatim}

    \begin{Verbatim}[commandchars=\\\{\}]
\{'OneTwo': 12, 'One': 1\}
    \end{Verbatim}

    That is how a dictionary looks like. Now you are able to access `1' by
the index value set at `One'

    \begin{Verbatim}[commandchars=\\\{\}]
{\color{incolor}In [{\color{incolor}32}]:} \PY{k}{print} \PY{n}{d0}\PY{p}{[}\PY{l+s}{\PYZsq{}}\PY{l+s}{One}\PY{l+s}{\PYZsq{}}\PY{p}{]}
\end{Verbatim}

    \begin{Verbatim}[commandchars=\\\{\}]
1
    \end{Verbatim}

    Two lists which are related can be merged to form a dictionary.

    \begin{Verbatim}[commandchars=\\\{\}]
{\color{incolor}In [{\color{incolor}33}]:} \PY{n}{names} \PY{o}{=} \PY{p}{[}\PY{l+s}{\PYZsq{}}\PY{l+s}{One}\PY{l+s}{\PYZsq{}}\PY{p}{,} \PY{l+s}{\PYZsq{}}\PY{l+s}{Two}\PY{l+s}{\PYZsq{}}\PY{p}{,} \PY{l+s}{\PYZsq{}}\PY{l+s}{Three}\PY{l+s}{\PYZsq{}}\PY{p}{,} \PY{l+s}{\PYZsq{}}\PY{l+s}{Four}\PY{l+s}{\PYZsq{}}\PY{p}{,} \PY{l+s}{\PYZsq{}}\PY{l+s}{Five}\PY{l+s}{\PYZsq{}}\PY{p}{]}
         \PY{n}{numbers} \PY{o}{=} \PY{p}{[}\PY{l+m+mi}{1}\PY{p}{,} \PY{l+m+mi}{2}\PY{p}{,} \PY{l+m+mi}{3}\PY{p}{,} \PY{l+m+mi}{4}\PY{p}{,} \PY{l+m+mi}{5}\PY{p}{]}
\end{Verbatim}

    \textbf{zip( )} function is used to combine two lists

    \begin{Verbatim}[commandchars=\\\{\}]
{\color{incolor}In [{\color{incolor}34}]:} \PY{n}{d2} \PY{o}{=} \PY{n+nb}{zip}\PY{p}{(}\PY{n}{names}\PY{p}{,}\PY{n}{numbers}\PY{p}{)}
         \PY{k}{print} \PY{n}{d2}
\end{Verbatim}

    \begin{Verbatim}[commandchars=\\\{\}]
[('One', 1), ('Two', 2), ('Three', 3), ('Four', 4), ('Five', 5)]
    \end{Verbatim}

    The two lists are combined to form a single list and each elements are
clubbed with their respective elements from the other list inside a
tuple. Tuples because that is what is assigned and the value should not
change.

Further, To convert the above into a dictionary. \textbf{dict( )}
function is used.

    \begin{Verbatim}[commandchars=\\\{\}]
{\color{incolor}In [{\color{incolor}35}]:} \PY{n}{a1} \PY{o}{=} \PY{n+nb}{dict}\PY{p}{(}\PY{n}{d2}\PY{p}{)}
         \PY{k}{print} \PY{n}{a1}
\end{Verbatim}

    \begin{Verbatim}[commandchars=\\\{\}]
\{'Four': 4, 'Five': 5, 'Three': 3, 'Two': 2, 'One': 1\}
    \end{Verbatim}

    \subsubsection{Built-in Functions}\label{built-in-functions}

    \textbf{clear( )} function is used to erase the entire database that was
created.

    \begin{Verbatim}[commandchars=\\\{\}]
{\color{incolor}In [{\color{incolor}36}]:} \PY{n}{a1}\PY{o}{.}\PY{n}{clear}\PY{p}{(}\PY{p}{)}
         \PY{k}{print} \PY{n}{a1}
\end{Verbatim}

    \begin{Verbatim}[commandchars=\\\{\}]
\{\}
    \end{Verbatim}

    Dictionary can also be built using loops.

    \begin{Verbatim}[commandchars=\\\{\}]
{\color{incolor}In [{\color{incolor}37}]:} \PY{k}{for} \PY{n}{i} \PY{o+ow}{in} \PY{n+nb}{range}\PY{p}{(}\PY{n+nb}{len}\PY{p}{(}\PY{n}{names}\PY{p}{)}\PY{p}{)}\PY{p}{:}
             \PY{n}{a1}\PY{p}{[}\PY{n}{names}\PY{p}{[}\PY{n}{i}\PY{p}{]}\PY{p}{]} \PY{o}{=} \PY{n}{numbers}\PY{p}{[}\PY{n}{i}\PY{p}{]}
         \PY{k}{print} \PY{n}{a1}
\end{Verbatim}

    \begin{Verbatim}[commandchars=\\\{\}]
\{'Four': 4, 'Five': 5, 'Three': 3, 'Two': 2, 'One': 1\}
    \end{Verbatim}

    \textbf{values( )} function returns a list with all the assigned values
in the dictionary.

    \begin{Verbatim}[commandchars=\\\{\}]
{\color{incolor}In [{\color{incolor}38}]:} \PY{n}{a1}\PY{o}{.}\PY{n}{values}\PY{p}{(}\PY{p}{)}
\end{Verbatim}

            \begin{Verbatim}[commandchars=\\\{\}]
{\color{outcolor}Out[{\color{outcolor}38}]:} [4, 5, 3, 2, 1]
\end{Verbatim}
        
    \textbf{keys( )} function returns all the index or the keys to which
contains the values that it was assigned to.

    \begin{Verbatim}[commandchars=\\\{\}]
{\color{incolor}In [{\color{incolor}39}]:} \PY{n}{a1}\PY{o}{.}\PY{n}{keys}\PY{p}{(}\PY{p}{)}
\end{Verbatim}

            \begin{Verbatim}[commandchars=\\\{\}]
{\color{outcolor}Out[{\color{outcolor}39}]:} ['Four', 'Five', 'Three', 'Two', 'One']
\end{Verbatim}
        
    \textbf{items( )} is returns a list containing both the list but each
element in the dictionary is inside a tuple. This is same as the result
that was obtained when zip function was used.

    \begin{Verbatim}[commandchars=\\\{\}]
{\color{incolor}In [{\color{incolor}40}]:} \PY{n}{a1}\PY{o}{.}\PY{n}{items}\PY{p}{(}\PY{p}{)}
\end{Verbatim}

            \begin{Verbatim}[commandchars=\\\{\}]
{\color{outcolor}Out[{\color{outcolor}40}]:} [('Four', 4), ('Five', 5), ('Three', 3), ('Two', 2), ('One', 1)]
\end{Verbatim}
        
    \textbf{pop( )} function is used to get the remove that particular
element and this removed element can be assigned to a new variable. But
remember only the value is stored and not the key. Because the is just a
index value.

    \begin{Verbatim}[commandchars=\\\{\}]
{\color{incolor}In [{\color{incolor}41}]:} \PY{n}{a2} \PY{o}{=} \PY{n}{a1}\PY{o}{.}\PY{n}{pop}\PY{p}{(}\PY{l+s}{\PYZsq{}}\PY{l+s}{Four}\PY{l+s}{\PYZsq{}}\PY{p}{)}
         \PY{k}{print} \PY{n}{a1}
         \PY{k}{print} \PY{n}{a2}
\end{Verbatim}

    \begin{Verbatim}[commandchars=\\\{\}]
\{'Five': 5, 'Three': 3, 'Two': 2, 'One': 1\}
4
    \end{Verbatim}


    % Add a bibliography block to the postdoc
    
  \newpage
  
  
% Default to the notebook output style

    


% Inherit from the specified cell style.




    
    
    
    \definecolor{orange}{cmyk}{0,0.4,0.8,0.2}
    \definecolor{darkorange}{rgb}{.71,0.21,0.01}
    \definecolor{darkgreen}{rgb}{.12,.54,.11}
    \definecolor{myteal}{rgb}{.26, .44, .56}
    \definecolor{gray}{gray}{0.45}
    \definecolor{lightgray}{gray}{.95}
    \definecolor{mediumgray}{gray}{.8}
    \definecolor{inputbackground}{rgb}{.95, .95, .85}
    \definecolor{outputbackground}{rgb}{.95, .95, .95}
    \definecolor{traceback}{rgb}{1, .95, .95}
    % ansi colors
    \definecolor{red}{rgb}{.6,0,0}
    \definecolor{green}{rgb}{0,.65,0}
    \definecolor{brown}{rgb}{0.6,0.6,0}
    \definecolor{blue}{rgb}{0,.145,.698}
    \definecolor{purple}{rgb}{.698,.145,.698}
    \definecolor{cyan}{rgb}{0,.698,.698}
    \definecolor{lightgray}{gray}{0.5}
    
    % bright ansi colors
    \definecolor{darkgray}{gray}{0.25}
    \definecolor{lightred}{rgb}{1.0,0.39,0.28}
    \definecolor{lightgreen}{rgb}{0.48,0.99,0.0}
    \definecolor{lightblue}{rgb}{0.53,0.81,0.92}
    \definecolor{lightpurple}{rgb}{0.87,0.63,0.87}
    \definecolor{lightcyan}{rgb}{0.5,1.0,0.83}
    
    % commands and environments needed by pandoc snippets
    % extracted from the output of `pandoc -s`
    \DefineVerbatimEnvironment{Highlighting}{Verbatim}{commandchars=\\\{\}}
    % Add ',fontsize=\small' for more characters per line
    \newenvironment{Shaded}{}{}
    \newcommand{\KeywordTok}[1]{\textcolor[rgb]{0.00,0.44,0.13}{\textbf{{#1}}}}
    \newcommand{\DataTypeTok}[1]{\textcolor[rgb]{0.56,0.13,0.00}{{#1}}}
    \newcommand{\DecValTok}[1]{\textcolor[rgb]{0.25,0.63,0.44}{{#1}}}
    \newcommand{\BaseNTok}[1]{\textcolor[rgb]{0.25,0.63,0.44}{{#1}}}
    \newcommand{\FloatTok}[1]{\textcolor[rgb]{0.25,0.63,0.44}{{#1}}}
    \newcommand{\CharTok}[1]{\textcolor[rgb]{0.25,0.44,0.63}{{#1}}}
    \newcommand{\StringTok}[1]{\textcolor[rgb]{0.25,0.44,0.63}{{#1}}}
    \newcommand{\CommentTok}[1]{\textcolor[rgb]{0.38,0.63,0.69}{\textit{{#1}}}}
    \newcommand{\OtherTok}[1]{\textcolor[rgb]{0.00,0.44,0.13}{{#1}}}
    \newcommand{\AlertTok}[1]{\textcolor[rgb]{1.00,0.00,0.00}{\textbf{{#1}}}}
    \newcommand{\FunctionTok}[1]{\textcolor[rgb]{0.02,0.16,0.49}{{#1}}}
    \newcommand{\RegionMarkerTok}[1]{{#1}}
    \newcommand{\ErrorTok}[1]{\textcolor[rgb]{1.00,0.00,0.00}{\textbf{{#1}}}}
    \newcommand{\NormalTok}[1]{{#1}}
    
    % Define a nice break command that doesn't care if a line doesn't already
    % exist.
    \def\br{\hspace*{\fill} \\* }
    % Math Jax compatability definitions
    \def\gt{>}
    \def\lt{<}
    % Document parameters
    \title{}
    
    
    

    % Pygments definitions
    
\makeatletter
\def\PY@reset{\let\PY@it=\relax \let\PY@bf=\relax%
    \let\PY@ul=\relax \let\PY@tc=\relax%
    \let\PY@bc=\relax \let\PY@ff=\relax}
\def\PY@tok#1{\csname PY@tok@#1\endcsname}
\def\PY@toks#1+{\ifx\relax#1\empty\else%
    \PY@tok{#1}\expandafter\PY@toks\fi}
\def\PY@do#1{\PY@bc{\PY@tc{\PY@ul{%
    \PY@it{\PY@bf{\PY@ff{#1}}}}}}}
\def\PY#1#2{\PY@reset\PY@toks#1+\relax+\PY@do{#2}}

\expandafter\def\csname PY@tok@gd\endcsname{\def\PY@tc##1{\textcolor[rgb]{0.63,0.00,0.00}{##1}}}
\expandafter\def\csname PY@tok@gu\endcsname{\let\PY@bf=\textbf\def\PY@tc##1{\textcolor[rgb]{0.50,0.00,0.50}{##1}}}
\expandafter\def\csname PY@tok@gt\endcsname{\def\PY@tc##1{\textcolor[rgb]{0.00,0.27,0.87}{##1}}}
\expandafter\def\csname PY@tok@gs\endcsname{\let\PY@bf=\textbf}
\expandafter\def\csname PY@tok@gr\endcsname{\def\PY@tc##1{\textcolor[rgb]{1.00,0.00,0.00}{##1}}}
\expandafter\def\csname PY@tok@cm\endcsname{\let\PY@it=\textit\def\PY@tc##1{\textcolor[rgb]{0.25,0.50,0.50}{##1}}}
\expandafter\def\csname PY@tok@vg\endcsname{\def\PY@tc##1{\textcolor[rgb]{0.10,0.09,0.49}{##1}}}
\expandafter\def\csname PY@tok@m\endcsname{\def\PY@tc##1{\textcolor[rgb]{0.40,0.40,0.40}{##1}}}
\expandafter\def\csname PY@tok@mh\endcsname{\def\PY@tc##1{\textcolor[rgb]{0.40,0.40,0.40}{##1}}}
\expandafter\def\csname PY@tok@go\endcsname{\def\PY@tc##1{\textcolor[rgb]{0.53,0.53,0.53}{##1}}}
\expandafter\def\csname PY@tok@ge\endcsname{\let\PY@it=\textit}
\expandafter\def\csname PY@tok@vc\endcsname{\def\PY@tc##1{\textcolor[rgb]{0.10,0.09,0.49}{##1}}}
\expandafter\def\csname PY@tok@il\endcsname{\def\PY@tc##1{\textcolor[rgb]{0.40,0.40,0.40}{##1}}}
\expandafter\def\csname PY@tok@cs\endcsname{\let\PY@it=\textit\def\PY@tc##1{\textcolor[rgb]{0.25,0.50,0.50}{##1}}}
\expandafter\def\csname PY@tok@cp\endcsname{\def\PY@tc##1{\textcolor[rgb]{0.74,0.48,0.00}{##1}}}
\expandafter\def\csname PY@tok@gi\endcsname{\def\PY@tc##1{\textcolor[rgb]{0.00,0.63,0.00}{##1}}}
\expandafter\def\csname PY@tok@gh\endcsname{\let\PY@bf=\textbf\def\PY@tc##1{\textcolor[rgb]{0.00,0.00,0.50}{##1}}}
\expandafter\def\csname PY@tok@ni\endcsname{\let\PY@bf=\textbf\def\PY@tc##1{\textcolor[rgb]{0.60,0.60,0.60}{##1}}}
\expandafter\def\csname PY@tok@nl\endcsname{\def\PY@tc##1{\textcolor[rgb]{0.63,0.63,0.00}{##1}}}
\expandafter\def\csname PY@tok@nn\endcsname{\let\PY@bf=\textbf\def\PY@tc##1{\textcolor[rgb]{0.00,0.00,1.00}{##1}}}
\expandafter\def\csname PY@tok@no\endcsname{\def\PY@tc##1{\textcolor[rgb]{0.53,0.00,0.00}{##1}}}
\expandafter\def\csname PY@tok@na\endcsname{\def\PY@tc##1{\textcolor[rgb]{0.49,0.56,0.16}{##1}}}
\expandafter\def\csname PY@tok@nb\endcsname{\def\PY@tc##1{\textcolor[rgb]{0.00,0.50,0.00}{##1}}}
\expandafter\def\csname PY@tok@nc\endcsname{\let\PY@bf=\textbf\def\PY@tc##1{\textcolor[rgb]{0.00,0.00,1.00}{##1}}}
\expandafter\def\csname PY@tok@nd\endcsname{\def\PY@tc##1{\textcolor[rgb]{0.67,0.13,1.00}{##1}}}
\expandafter\def\csname PY@tok@ne\endcsname{\let\PY@bf=\textbf\def\PY@tc##1{\textcolor[rgb]{0.82,0.25,0.23}{##1}}}
\expandafter\def\csname PY@tok@nf\endcsname{\def\PY@tc##1{\textcolor[rgb]{0.00,0.00,1.00}{##1}}}
\expandafter\def\csname PY@tok@si\endcsname{\let\PY@bf=\textbf\def\PY@tc##1{\textcolor[rgb]{0.73,0.40,0.53}{##1}}}
\expandafter\def\csname PY@tok@s2\endcsname{\def\PY@tc##1{\textcolor[rgb]{0.73,0.13,0.13}{##1}}}
\expandafter\def\csname PY@tok@vi\endcsname{\def\PY@tc##1{\textcolor[rgb]{0.10,0.09,0.49}{##1}}}
\expandafter\def\csname PY@tok@nt\endcsname{\let\PY@bf=\textbf\def\PY@tc##1{\textcolor[rgb]{0.00,0.50,0.00}{##1}}}
\expandafter\def\csname PY@tok@nv\endcsname{\def\PY@tc##1{\textcolor[rgb]{0.10,0.09,0.49}{##1}}}
\expandafter\def\csname PY@tok@s1\endcsname{\def\PY@tc##1{\textcolor[rgb]{0.73,0.13,0.13}{##1}}}
\expandafter\def\csname PY@tok@kd\endcsname{\let\PY@bf=\textbf\def\PY@tc##1{\textcolor[rgb]{0.00,0.50,0.00}{##1}}}
\expandafter\def\csname PY@tok@sh\endcsname{\def\PY@tc##1{\textcolor[rgb]{0.73,0.13,0.13}{##1}}}
\expandafter\def\csname PY@tok@sc\endcsname{\def\PY@tc##1{\textcolor[rgb]{0.73,0.13,0.13}{##1}}}
\expandafter\def\csname PY@tok@sx\endcsname{\def\PY@tc##1{\textcolor[rgb]{0.00,0.50,0.00}{##1}}}
\expandafter\def\csname PY@tok@bp\endcsname{\def\PY@tc##1{\textcolor[rgb]{0.00,0.50,0.00}{##1}}}
\expandafter\def\csname PY@tok@c1\endcsname{\let\PY@it=\textit\def\PY@tc##1{\textcolor[rgb]{0.25,0.50,0.50}{##1}}}
\expandafter\def\csname PY@tok@kc\endcsname{\let\PY@bf=\textbf\def\PY@tc##1{\textcolor[rgb]{0.00,0.50,0.00}{##1}}}
\expandafter\def\csname PY@tok@c\endcsname{\let\PY@it=\textit\def\PY@tc##1{\textcolor[rgb]{0.25,0.50,0.50}{##1}}}
\expandafter\def\csname PY@tok@mf\endcsname{\def\PY@tc##1{\textcolor[rgb]{0.40,0.40,0.40}{##1}}}
\expandafter\def\csname PY@tok@err\endcsname{\def\PY@bc##1{\setlength{\fboxsep}{0pt}\fcolorbox[rgb]{1.00,0.00,0.00}{1,1,1}{\strut ##1}}}
\expandafter\def\csname PY@tok@mb\endcsname{\def\PY@tc##1{\textcolor[rgb]{0.40,0.40,0.40}{##1}}}
\expandafter\def\csname PY@tok@ss\endcsname{\def\PY@tc##1{\textcolor[rgb]{0.10,0.09,0.49}{##1}}}
\expandafter\def\csname PY@tok@sr\endcsname{\def\PY@tc##1{\textcolor[rgb]{0.73,0.40,0.53}{##1}}}
\expandafter\def\csname PY@tok@mo\endcsname{\def\PY@tc##1{\textcolor[rgb]{0.40,0.40,0.40}{##1}}}
\expandafter\def\csname PY@tok@kn\endcsname{\let\PY@bf=\textbf\def\PY@tc##1{\textcolor[rgb]{0.00,0.50,0.00}{##1}}}
\expandafter\def\csname PY@tok@mi\endcsname{\def\PY@tc##1{\textcolor[rgb]{0.40,0.40,0.40}{##1}}}
\expandafter\def\csname PY@tok@gp\endcsname{\let\PY@bf=\textbf\def\PY@tc##1{\textcolor[rgb]{0.00,0.00,0.50}{##1}}}
\expandafter\def\csname PY@tok@o\endcsname{\def\PY@tc##1{\textcolor[rgb]{0.40,0.40,0.40}{##1}}}
\expandafter\def\csname PY@tok@kr\endcsname{\let\PY@bf=\textbf\def\PY@tc##1{\textcolor[rgb]{0.00,0.50,0.00}{##1}}}
\expandafter\def\csname PY@tok@s\endcsname{\def\PY@tc##1{\textcolor[rgb]{0.73,0.13,0.13}{##1}}}
\expandafter\def\csname PY@tok@kp\endcsname{\def\PY@tc##1{\textcolor[rgb]{0.00,0.50,0.00}{##1}}}
\expandafter\def\csname PY@tok@w\endcsname{\def\PY@tc##1{\textcolor[rgb]{0.73,0.73,0.73}{##1}}}
\expandafter\def\csname PY@tok@kt\endcsname{\def\PY@tc##1{\textcolor[rgb]{0.69,0.00,0.25}{##1}}}
\expandafter\def\csname PY@tok@ow\endcsname{\let\PY@bf=\textbf\def\PY@tc##1{\textcolor[rgb]{0.67,0.13,1.00}{##1}}}
\expandafter\def\csname PY@tok@sb\endcsname{\def\PY@tc##1{\textcolor[rgb]{0.73,0.13,0.13}{##1}}}
\expandafter\def\csname PY@tok@k\endcsname{\let\PY@bf=\textbf\def\PY@tc##1{\textcolor[rgb]{0.00,0.50,0.00}{##1}}}
\expandafter\def\csname PY@tok@se\endcsname{\let\PY@bf=\textbf\def\PY@tc##1{\textcolor[rgb]{0.73,0.40,0.13}{##1}}}
\expandafter\def\csname PY@tok@sd\endcsname{\let\PY@it=\textit\def\PY@tc##1{\textcolor[rgb]{0.73,0.13,0.13}{##1}}}

\def\PYZbs{\char`\\}
\def\PYZus{\char`\_}
\def\PYZob{\char`\{}
\def\PYZcb{\char`\}}
\def\PYZca{\char`\^}
\def\PYZam{\char`\&}
\def\PYZlt{\char`\<}
\def\PYZgt{\char`\>}
\def\PYZsh{\char`\#}
\def\PYZpc{\char`\%}
\def\PYZdl{\char`\$}
\def\PYZhy{\char`\-}
\def\PYZsq{\char`\'}
\def\PYZdq{\char`\"}
\def\PYZti{\char`\~}
% for compatibility with earlier versions
\def\PYZat{@}
\def\PYZlb{[}
\def\PYZrb{]}
\makeatother


    % Exact colors from NB
    \definecolor{incolor}{rgb}{0.0, 0.0, 0.5}
    \definecolor{outcolor}{rgb}{0.545, 0.0, 0.0}



    
    % Prevent overflowing lines due to hard-to-break entities
    \sloppy 
    % Setup hyperref package
    \hypersetup{
      breaklinks=true,  % so long urls are correctly broken across lines
      colorlinks=true,
      urlcolor=blue,
      linkcolor=darkorange,
      citecolor=darkgreen,
      }
    % Slightly bigger margins than the latex defaults
    
    
    

    \begin{document}
    
    
    \maketitle
    
    

    
    \section{Control Flow Statements}\label{control-flow-statements}

    \subsection{If}\label{if}

    if some\_condition:

\begin{verbatim}
algorithm
\end{verbatim}

    \begin{Verbatim}[commandchars=\\\{\}]
{\color{incolor}In [{\color{incolor}1}]:} \PY{n}{x} \PY{o}{=} \PY{l+m+mi}{12}
        \PY{k}{if} \PY{n}{x} \PY{o}{\PYZgt{}}\PY{l+m+mi}{10}\PY{p}{:}
            \PY{k}{print} \PY{l+s}{\PYZdq{}}\PY{l+s}{Hello}\PY{l+s}{\PYZdq{}}
\end{Verbatim}

    \begin{Verbatim}[commandchars=\\\{\}]
Hello
    \end{Verbatim}

    \subsection{If-else}\label{if-else}

    if some\_condition:

\begin{verbatim}
algorithm
\end{verbatim}

else:

\begin{verbatim}
algorithm
\end{verbatim}

    \begin{Verbatim}[commandchars=\\\{\}]
{\color{incolor}In [{\color{incolor}2}]:} \PY{n}{x} \PY{o}{=} \PY{l+m+mi}{12}
        \PY{k}{if} \PY{n}{x} \PY{o}{\PYZgt{}} \PY{l+m+mi}{10}\PY{p}{:}
            \PY{k}{print} \PY{l+s}{\PYZdq{}}\PY{l+s}{hello}\PY{l+s}{\PYZdq{}}
        \PY{k}{else}\PY{p}{:}
            \PY{k}{print} \PY{l+s}{\PYZdq{}}\PY{l+s}{world}\PY{l+s}{\PYZdq{}}
\end{Verbatim}

    \begin{Verbatim}[commandchars=\\\{\}]
hello
    \end{Verbatim}

    \subsection{if-elif}\label{if-elif}

    if some\_condition:

\begin{verbatim}
algorithm
\end{verbatim}

elif some\_condition:

\begin{verbatim}
algorithm
\end{verbatim}

else:

\begin{verbatim}
algorithm
\end{verbatim}

    \begin{Verbatim}[commandchars=\\\{\}]
{\color{incolor}In [{\color{incolor}3}]:} \PY{n}{x} \PY{o}{=} \PY{l+m+mi}{10}
        \PY{n}{y} \PY{o}{=} \PY{l+m+mi}{12}
        \PY{k}{if} \PY{n}{x} \PY{o}{\PYZgt{}} \PY{n}{y}\PY{p}{:}
            \PY{k}{print} \PY{l+s}{\PYZdq{}}\PY{l+s}{x\PYZgt{}y}\PY{l+s}{\PYZdq{}}
        \PY{k}{elif} \PY{n}{x} \PY{o}{\PYZlt{}} \PY{n}{y}\PY{p}{:}
            \PY{k}{print} \PY{l+s}{\PYZdq{}}\PY{l+s}{x\PYZlt{}y}\PY{l+s}{\PYZdq{}}
        \PY{k}{else}\PY{p}{:}
            \PY{k}{print} \PY{l+s}{\PYZdq{}}\PY{l+s}{x=y}\PY{l+s}{\PYZdq{}}
\end{Verbatim}

    \begin{Verbatim}[commandchars=\\\{\}]
x<y
    \end{Verbatim}

    if statement inside a if statement or if-elif or if-else are called as
nested if statements.

    \begin{Verbatim}[commandchars=\\\{\}]
{\color{incolor}In [{\color{incolor}4}]:} \PY{n}{x} \PY{o}{=} \PY{l+m+mi}{10}
        \PY{n}{y} \PY{o}{=} \PY{l+m+mi}{12}
        \PY{k}{if} \PY{n}{x} \PY{o}{\PYZgt{}} \PY{n}{y}\PY{p}{:}
            \PY{k}{print} \PY{l+s}{\PYZdq{}}\PY{l+s}{x\PYZgt{}y}\PY{l+s}{\PYZdq{}}
        \PY{k}{elif} \PY{n}{x} \PY{o}{\PYZlt{}} \PY{n}{y}\PY{p}{:}
            \PY{k}{print} \PY{l+s}{\PYZdq{}}\PY{l+s}{x\PYZlt{}y}\PY{l+s}{\PYZdq{}}
            \PY{k}{if} \PY{n}{x}\PY{o}{==}\PY{l+m+mi}{10}\PY{p}{:}
                \PY{k}{print} \PY{l+s}{\PYZdq{}}\PY{l+s}{x=10}\PY{l+s}{\PYZdq{}}
            \PY{k}{else}\PY{p}{:}
                \PY{k}{print} \PY{l+s}{\PYZdq{}}\PY{l+s}{invalid}\PY{l+s}{\PYZdq{}}
        \PY{k}{else}\PY{p}{:}
            \PY{k}{print} \PY{l+s}{\PYZdq{}}\PY{l+s}{x=y}\PY{l+s}{\PYZdq{}}
\end{Verbatim}

    \begin{Verbatim}[commandchars=\\\{\}]
x<y
x=10
    \end{Verbatim}

    \subsection{Loops}\label{loops}

    \subsubsection{For}\label{for}

    for variable in something:

\begin{verbatim}
algorithm
\end{verbatim}

    \begin{Verbatim}[commandchars=\\\{\}]
{\color{incolor}In [{\color{incolor}5}]:} \PY{k}{for} \PY{n}{i} \PY{o+ow}{in} \PY{n+nb}{range}\PY{p}{(}\PY{l+m+mi}{5}\PY{p}{)}\PY{p}{:}
            \PY{k}{print} \PY{n}{i}
\end{Verbatim}

    \begin{Verbatim}[commandchars=\\\{\}]
0
1
2
3
4
    \end{Verbatim}

    In the above example, i iterates over the 0,1,2,3,4. Every time it takes
each value and executes the algorithm inside the loop. It is also
possible to iterate over a nested list illustrated below.

    \begin{Verbatim}[commandchars=\\\{\}]
{\color{incolor}In [{\color{incolor}6}]:} \PY{n}{list\PYZus{}of\PYZus{}lists} \PY{o}{=} \PY{p}{[}\PY{p}{[}\PY{l+m+mi}{1}\PY{p}{,} \PY{l+m+mi}{2}\PY{p}{,} \PY{l+m+mi}{3}\PY{p}{]}\PY{p}{,} \PY{p}{[}\PY{l+m+mi}{4}\PY{p}{,} \PY{l+m+mi}{5}\PY{p}{,} \PY{l+m+mi}{6}\PY{p}{]}\PY{p}{,} \PY{p}{[}\PY{l+m+mi}{7}\PY{p}{,} \PY{l+m+mi}{8}\PY{p}{,} \PY{l+m+mi}{9}\PY{p}{]}\PY{p}{]}
        \PY{k}{for} \PY{n}{list1} \PY{o+ow}{in} \PY{n}{list\PYZus{}of\PYZus{}lists}\PY{p}{:}
                \PY{k}{print} \PY{n}{list1}
\end{Verbatim}

    \begin{Verbatim}[commandchars=\\\{\}]
[1, 2, 3]
[4, 5, 6]
[7, 8, 9]
    \end{Verbatim}

    A use case of a nested for loop in this case would be,

    \begin{Verbatim}[commandchars=\\\{\}]
{\color{incolor}In [{\color{incolor}7}]:} \PY{n}{list\PYZus{}of\PYZus{}lists} \PY{o}{=} \PY{p}{[}\PY{p}{[}\PY{l+m+mi}{1}\PY{p}{,} \PY{l+m+mi}{2}\PY{p}{,} \PY{l+m+mi}{3}\PY{p}{]}\PY{p}{,} \PY{p}{[}\PY{l+m+mi}{4}\PY{p}{,} \PY{l+m+mi}{5}\PY{p}{,} \PY{l+m+mi}{6}\PY{p}{]}\PY{p}{,} \PY{p}{[}\PY{l+m+mi}{7}\PY{p}{,} \PY{l+m+mi}{8}\PY{p}{,} \PY{l+m+mi}{9}\PY{p}{]}\PY{p}{]}
        \PY{k}{for} \PY{n}{list1} \PY{o+ow}{in} \PY{n}{list\PYZus{}of\PYZus{}lists}\PY{p}{:}
            \PY{k}{for} \PY{n}{x} \PY{o+ow}{in} \PY{n}{list1}\PY{p}{:}
                \PY{k}{print} \PY{n}{x}
\end{Verbatim}

    \begin{Verbatim}[commandchars=\\\{\}]
1
2
3
4
5
6
7
8
9
    \end{Verbatim}

    \subsubsection{While}\label{while}

    while some\_condition:

\begin{verbatim}
algorithm
\end{verbatim}

    \begin{Verbatim}[commandchars=\\\{\}]
{\color{incolor}In [{\color{incolor}8}]:} \PY{n}{i} \PY{o}{=} \PY{l+m+mi}{1}
        \PY{k}{while} \PY{n}{i} \PY{o}{\PYZlt{}} \PY{l+m+mi}{3}\PY{p}{:}
            \PY{k}{print}\PY{p}{(}\PY{n}{i} \PY{o}{*}\PY{o}{*} \PY{l+m+mi}{2}\PY{p}{)}
            \PY{n}{i} \PY{o}{=} \PY{n}{i}\PY{o}{+}\PY{l+m+mi}{1}
        \PY{k}{print}\PY{p}{(}\PY{l+s}{\PYZsq{}}\PY{l+s}{Bye}\PY{l+s}{\PYZsq{}}\PY{p}{)}
\end{Verbatim}

    \begin{Verbatim}[commandchars=\\\{\}]
1
4
Bye
    \end{Verbatim}

    \subsection{Break}\label{break}

    As the name says. It is used to break out of a loop when a condition
becomes true when executing the loop.

    \begin{Verbatim}[commandchars=\\\{\}]
{\color{incolor}In [{\color{incolor}9}]:} \PY{k}{for} \PY{n}{i} \PY{o+ow}{in} \PY{n+nb}{range}\PY{p}{(}\PY{l+m+mi}{100}\PY{p}{)}\PY{p}{:}
            \PY{k}{print} \PY{n}{i}
            \PY{k}{if} \PY{n}{i}\PY{o}{\PYZgt{}}\PY{o}{=}\PY{l+m+mi}{7}\PY{p}{:}
                \PY{k}{break}
\end{Verbatim}

    \begin{Verbatim}[commandchars=\\\{\}]
0
1
2
3
4
5
6
7
    \end{Verbatim}

    \subsection{Continue}\label{continue}

    This continues the rest of the loop. Sometimes when a condition is
satisfied there are chances of the loop getting terminated. This can be
avoided using continue statement.

    \begin{Verbatim}[commandchars=\\\{\}]
{\color{incolor}In [{\color{incolor}10}]:} \PY{k}{for} \PY{n}{i} \PY{o+ow}{in} \PY{n+nb}{range}\PY{p}{(}\PY{l+m+mi}{10}\PY{p}{)}\PY{p}{:}
             \PY{k}{if} \PY{n}{i}\PY{o}{\PYZgt{}}\PY{l+m+mi}{4}\PY{p}{:}
                 \PY{k}{print} \PY{l+s}{\PYZdq{}}\PY{l+s}{The end.}\PY{l+s}{\PYZdq{}}
                 \PY{k}{continue}
             \PY{k}{elif} \PY{n}{i}\PY{o}{\PYZlt{}}\PY{l+m+mi}{7}\PY{p}{:}
                 \PY{k}{print} \PY{n}{i}
\end{Verbatim}

    \begin{Verbatim}[commandchars=\\\{\}]
0
1
2
3
4
The end.
The end.
The end.
The end.
The end.
    \end{Verbatim}

    \subsection{List Comprehensions}\label{list-comprehensions}

    Python makes it simple to generate a required list with a single line of
code using list comprehensions. For example If i need to generate
multiples of say 27 I write the code using for loop as,

    \begin{Verbatim}[commandchars=\\\{\}]
{\color{incolor}In [{\color{incolor}11}]:} \PY{n}{res} \PY{o}{=} \PY{p}{[}\PY{p}{]}
         \PY{k}{for} \PY{n}{i} \PY{o+ow}{in} \PY{n+nb}{range}\PY{p}{(}\PY{l+m+mi}{1}\PY{p}{,}\PY{l+m+mi}{11}\PY{p}{)}\PY{p}{:}
             \PY{n}{x} \PY{o}{=} \PY{l+m+mi}{27}\PY{o}{*}\PY{n}{i}
             \PY{n}{res}\PY{o}{.}\PY{n}{append}\PY{p}{(}\PY{n}{x}\PY{p}{)}
         \PY{k}{print} \PY{n}{res}
\end{Verbatim}

    \begin{Verbatim}[commandchars=\\\{\}]
[27, 54, 81, 108, 135, 162, 189, 216, 243, 270]
    \end{Verbatim}

    Since you are generating another list altogether and that is what is
required, List comprehensions is a more efficient way to solve this
problem.

    \begin{Verbatim}[commandchars=\\\{\}]
{\color{incolor}In [{\color{incolor}12}]:} \PY{p}{[}\PY{l+m+mi}{27}\PY{o}{*}\PY{n}{x} \PY{k}{for} \PY{n}{x} \PY{o+ow}{in} \PY{n+nb}{range}\PY{p}{(}\PY{l+m+mi}{1}\PY{p}{,}\PY{l+m+mi}{11}\PY{p}{)}\PY{p}{]}
\end{Verbatim}

            \begin{Verbatim}[commandchars=\\\{\}]
{\color{outcolor}Out[{\color{outcolor}12}]:} [27, 54, 81, 108, 135, 162, 189, 216, 243, 270]
\end{Verbatim}
        
    That's it!. Only remember to enclose it in square brackets

    Understanding the code, The first bit of the code is always the
algorithm and then leave a space and then write the necessary loop. But
you might be wondering can nested loops be extended to list
comprehensions? Yes you can.

    \begin{Verbatim}[commandchars=\\\{\}]
{\color{incolor}In [{\color{incolor}13}]:} \PY{p}{[}\PY{l+m+mi}{27}\PY{o}{*}\PY{n}{x} \PY{k}{for} \PY{n}{x} \PY{o+ow}{in} \PY{n+nb}{range}\PY{p}{(}\PY{l+m+mi}{1}\PY{p}{,}\PY{l+m+mi}{20}\PY{p}{)} \PY{k}{if} \PY{n}{x}\PY{o}{\PYZlt{}}\PY{o}{=}\PY{l+m+mi}{10}\PY{p}{]}
\end{Verbatim}

            \begin{Verbatim}[commandchars=\\\{\}]
{\color{outcolor}Out[{\color{outcolor}13}]:} [27, 54, 81, 108, 135, 162, 189, 216, 243, 270]
\end{Verbatim}
        
    Let me add one more loop to make you understand better,

    \begin{Verbatim}[commandchars=\\\{\}]
{\color{incolor}In [{\color{incolor}14}]:} \PY{p}{[}\PY{l+m+mi}{27}\PY{o}{*}\PY{n}{z} \PY{k}{for} \PY{n}{i} \PY{o+ow}{in} \PY{n+nb}{range}\PY{p}{(}\PY{l+m+mi}{50}\PY{p}{)} \PY{k}{if} \PY{n}{i}\PY{o}{==}\PY{l+m+mi}{27} \PY{k}{for} \PY{n}{z} \PY{o+ow}{in} \PY{n+nb}{range}\PY{p}{(}\PY{l+m+mi}{1}\PY{p}{,}\PY{l+m+mi}{11}\PY{p}{)}\PY{p}{]}
\end{Verbatim}

            \begin{Verbatim}[commandchars=\\\{\}]
{\color{outcolor}Out[{\color{outcolor}14}]:} [27, 54, 81, 108, 135, 162, 189, 216, 243, 270]
\end{Verbatim}
        

    % Add a bibliography block to the postdoc
  \newpage
  
  
% Default to the notebook output style

    


% Inherit from the specified cell style.




    
    
    
    \definecolor{orange}{cmyk}{0,0.4,0.8,0.2}
    \definecolor{darkorange}{rgb}{.71,0.21,0.01}
    \definecolor{darkgreen}{rgb}{.12,.54,.11}
    \definecolor{myteal}{rgb}{.26, .44, .56}
    \definecolor{gray}{gray}{0.45}
    \definecolor{lightgray}{gray}{.95}
    \definecolor{mediumgray}{gray}{.8}
    \definecolor{inputbackground}{rgb}{.95, .95, .85}
    \definecolor{outputbackground}{rgb}{.95, .95, .95}
    \definecolor{traceback}{rgb}{1, .95, .95}
    % ansi colors
    \definecolor{red}{rgb}{.6,0,0}
    \definecolor{green}{rgb}{0,.65,0}
    \definecolor{brown}{rgb}{0.6,0.6,0}
    \definecolor{blue}{rgb}{0,.145,.698}
    \definecolor{purple}{rgb}{.698,.145,.698}
    \definecolor{cyan}{rgb}{0,.698,.698}
    \definecolor{lightgray}{gray}{0.5}
    
    % bright ansi colors
    \definecolor{darkgray}{gray}{0.25}
    \definecolor{lightred}{rgb}{1.0,0.39,0.28}
    \definecolor{lightgreen}{rgb}{0.48,0.99,0.0}
    \definecolor{lightblue}{rgb}{0.53,0.81,0.92}
    \definecolor{lightpurple}{rgb}{0.87,0.63,0.87}
    \definecolor{lightcyan}{rgb}{0.5,1.0,0.83}
    
    % commands and environments needed by pandoc snippets
    % extracted from the output of `pandoc -s`
    \DefineVerbatimEnvironment{Highlighting}{Verbatim}{commandchars=\\\{\}}
    % Add ',fontsize=\small' for more characters per line
    \newenvironment{Shaded}{}{}
    \newcommand{\KeywordTok}[1]{\textcolor[rgb]{0.00,0.44,0.13}{\textbf{{#1}}}}
    \newcommand{\DataTypeTok}[1]{\textcolor[rgb]{0.56,0.13,0.00}{{#1}}}
    \newcommand{\DecValTok}[1]{\textcolor[rgb]{0.25,0.63,0.44}{{#1}}}
    \newcommand{\BaseNTok}[1]{\textcolor[rgb]{0.25,0.63,0.44}{{#1}}}
    \newcommand{\FloatTok}[1]{\textcolor[rgb]{0.25,0.63,0.44}{{#1}}}
    \newcommand{\CharTok}[1]{\textcolor[rgb]{0.25,0.44,0.63}{{#1}}}
    \newcommand{\StringTok}[1]{\textcolor[rgb]{0.25,0.44,0.63}{{#1}}}
    \newcommand{\CommentTok}[1]{\textcolor[rgb]{0.38,0.63,0.69}{\textit{{#1}}}}
    \newcommand{\OtherTok}[1]{\textcolor[rgb]{0.00,0.44,0.13}{{#1}}}
    \newcommand{\AlertTok}[1]{\textcolor[rgb]{1.00,0.00,0.00}{\textbf{{#1}}}}
    \newcommand{\FunctionTok}[1]{\textcolor[rgb]{0.02,0.16,0.49}{{#1}}}
    \newcommand{\RegionMarkerTok}[1]{{#1}}
    \newcommand{\ErrorTok}[1]{\textcolor[rgb]{1.00,0.00,0.00}{\textbf{{#1}}}}
    \newcommand{\NormalTok}[1]{{#1}}
    
    % Define a nice break command that doesn't care if a line doesn't already
    % exist.
    \def\br{\hspace*{\fill} \\* }
    % Math Jax compatability definitions
    \def\gt{>}
    \def\lt{<}
    % Document parameters
    \title{}
    
    
    

    % Pygments definitions
    
\makeatletter
\def\PY@reset{\let\PY@it=\relax \let\PY@bf=\relax%
    \let\PY@ul=\relax \let\PY@tc=\relax%
    \let\PY@bc=\relax \let\PY@ff=\relax}
\def\PY@tok#1{\csname PY@tok@#1\endcsname}
\def\PY@toks#1+{\ifx\relax#1\empty\else%
    \PY@tok{#1}\expandafter\PY@toks\fi}
\def\PY@do#1{\PY@bc{\PY@tc{\PY@ul{%
    \PY@it{\PY@bf{\PY@ff{#1}}}}}}}
\def\PY#1#2{\PY@reset\PY@toks#1+\relax+\PY@do{#2}}

\expandafter\def\csname PY@tok@gd\endcsname{\def\PY@tc##1{\textcolor[rgb]{0.63,0.00,0.00}{##1}}}
\expandafter\def\csname PY@tok@gu\endcsname{\let\PY@bf=\textbf\def\PY@tc##1{\textcolor[rgb]{0.50,0.00,0.50}{##1}}}
\expandafter\def\csname PY@tok@gt\endcsname{\def\PY@tc##1{\textcolor[rgb]{0.00,0.27,0.87}{##1}}}
\expandafter\def\csname PY@tok@gs\endcsname{\let\PY@bf=\textbf}
\expandafter\def\csname PY@tok@gr\endcsname{\def\PY@tc##1{\textcolor[rgb]{1.00,0.00,0.00}{##1}}}
\expandafter\def\csname PY@tok@cm\endcsname{\let\PY@it=\textit\def\PY@tc##1{\textcolor[rgb]{0.25,0.50,0.50}{##1}}}
\expandafter\def\csname PY@tok@vg\endcsname{\def\PY@tc##1{\textcolor[rgb]{0.10,0.09,0.49}{##1}}}
\expandafter\def\csname PY@tok@m\endcsname{\def\PY@tc##1{\textcolor[rgb]{0.40,0.40,0.40}{##1}}}
\expandafter\def\csname PY@tok@mh\endcsname{\def\PY@tc##1{\textcolor[rgb]{0.40,0.40,0.40}{##1}}}
\expandafter\def\csname PY@tok@go\endcsname{\def\PY@tc##1{\textcolor[rgb]{0.53,0.53,0.53}{##1}}}
\expandafter\def\csname PY@tok@ge\endcsname{\let\PY@it=\textit}
\expandafter\def\csname PY@tok@vc\endcsname{\def\PY@tc##1{\textcolor[rgb]{0.10,0.09,0.49}{##1}}}
\expandafter\def\csname PY@tok@il\endcsname{\def\PY@tc##1{\textcolor[rgb]{0.40,0.40,0.40}{##1}}}
\expandafter\def\csname PY@tok@cs\endcsname{\let\PY@it=\textit\def\PY@tc##1{\textcolor[rgb]{0.25,0.50,0.50}{##1}}}
\expandafter\def\csname PY@tok@cp\endcsname{\def\PY@tc##1{\textcolor[rgb]{0.74,0.48,0.00}{##1}}}
\expandafter\def\csname PY@tok@gi\endcsname{\def\PY@tc##1{\textcolor[rgb]{0.00,0.63,0.00}{##1}}}
\expandafter\def\csname PY@tok@gh\endcsname{\let\PY@bf=\textbf\def\PY@tc##1{\textcolor[rgb]{0.00,0.00,0.50}{##1}}}
\expandafter\def\csname PY@tok@ni\endcsname{\let\PY@bf=\textbf\def\PY@tc##1{\textcolor[rgb]{0.60,0.60,0.60}{##1}}}
\expandafter\def\csname PY@tok@nl\endcsname{\def\PY@tc##1{\textcolor[rgb]{0.63,0.63,0.00}{##1}}}
\expandafter\def\csname PY@tok@nn\endcsname{\let\PY@bf=\textbf\def\PY@tc##1{\textcolor[rgb]{0.00,0.00,1.00}{##1}}}
\expandafter\def\csname PY@tok@no\endcsname{\def\PY@tc##1{\textcolor[rgb]{0.53,0.00,0.00}{##1}}}
\expandafter\def\csname PY@tok@na\endcsname{\def\PY@tc##1{\textcolor[rgb]{0.49,0.56,0.16}{##1}}}
\expandafter\def\csname PY@tok@nb\endcsname{\def\PY@tc##1{\textcolor[rgb]{0.00,0.50,0.00}{##1}}}
\expandafter\def\csname PY@tok@nc\endcsname{\let\PY@bf=\textbf\def\PY@tc##1{\textcolor[rgb]{0.00,0.00,1.00}{##1}}}
\expandafter\def\csname PY@tok@nd\endcsname{\def\PY@tc##1{\textcolor[rgb]{0.67,0.13,1.00}{##1}}}
\expandafter\def\csname PY@tok@ne\endcsname{\let\PY@bf=\textbf\def\PY@tc##1{\textcolor[rgb]{0.82,0.25,0.23}{##1}}}
\expandafter\def\csname PY@tok@nf\endcsname{\def\PY@tc##1{\textcolor[rgb]{0.00,0.00,1.00}{##1}}}
\expandafter\def\csname PY@tok@si\endcsname{\let\PY@bf=\textbf\def\PY@tc##1{\textcolor[rgb]{0.73,0.40,0.53}{##1}}}
\expandafter\def\csname PY@tok@s2\endcsname{\def\PY@tc##1{\textcolor[rgb]{0.73,0.13,0.13}{##1}}}
\expandafter\def\csname PY@tok@vi\endcsname{\def\PY@tc##1{\textcolor[rgb]{0.10,0.09,0.49}{##1}}}
\expandafter\def\csname PY@tok@nt\endcsname{\let\PY@bf=\textbf\def\PY@tc##1{\textcolor[rgb]{0.00,0.50,0.00}{##1}}}
\expandafter\def\csname PY@tok@nv\endcsname{\def\PY@tc##1{\textcolor[rgb]{0.10,0.09,0.49}{##1}}}
\expandafter\def\csname PY@tok@s1\endcsname{\def\PY@tc##1{\textcolor[rgb]{0.73,0.13,0.13}{##1}}}
\expandafter\def\csname PY@tok@kd\endcsname{\let\PY@bf=\textbf\def\PY@tc##1{\textcolor[rgb]{0.00,0.50,0.00}{##1}}}
\expandafter\def\csname PY@tok@sh\endcsname{\def\PY@tc##1{\textcolor[rgb]{0.73,0.13,0.13}{##1}}}
\expandafter\def\csname PY@tok@sc\endcsname{\def\PY@tc##1{\textcolor[rgb]{0.73,0.13,0.13}{##1}}}
\expandafter\def\csname PY@tok@sx\endcsname{\def\PY@tc##1{\textcolor[rgb]{0.00,0.50,0.00}{##1}}}
\expandafter\def\csname PY@tok@bp\endcsname{\def\PY@tc##1{\textcolor[rgb]{0.00,0.50,0.00}{##1}}}
\expandafter\def\csname PY@tok@c1\endcsname{\let\PY@it=\textit\def\PY@tc##1{\textcolor[rgb]{0.25,0.50,0.50}{##1}}}
\expandafter\def\csname PY@tok@kc\endcsname{\let\PY@bf=\textbf\def\PY@tc##1{\textcolor[rgb]{0.00,0.50,0.00}{##1}}}
\expandafter\def\csname PY@tok@c\endcsname{\let\PY@it=\textit\def\PY@tc##1{\textcolor[rgb]{0.25,0.50,0.50}{##1}}}
\expandafter\def\csname PY@tok@mf\endcsname{\def\PY@tc##1{\textcolor[rgb]{0.40,0.40,0.40}{##1}}}
\expandafter\def\csname PY@tok@err\endcsname{\def\PY@bc##1{\setlength{\fboxsep}{0pt}\fcolorbox[rgb]{1.00,0.00,0.00}{1,1,1}{\strut ##1}}}
\expandafter\def\csname PY@tok@mb\endcsname{\def\PY@tc##1{\textcolor[rgb]{0.40,0.40,0.40}{##1}}}
\expandafter\def\csname PY@tok@ss\endcsname{\def\PY@tc##1{\textcolor[rgb]{0.10,0.09,0.49}{##1}}}
\expandafter\def\csname PY@tok@sr\endcsname{\def\PY@tc##1{\textcolor[rgb]{0.73,0.40,0.53}{##1}}}
\expandafter\def\csname PY@tok@mo\endcsname{\def\PY@tc##1{\textcolor[rgb]{0.40,0.40,0.40}{##1}}}
\expandafter\def\csname PY@tok@kn\endcsname{\let\PY@bf=\textbf\def\PY@tc##1{\textcolor[rgb]{0.00,0.50,0.00}{##1}}}
\expandafter\def\csname PY@tok@mi\endcsname{\def\PY@tc##1{\textcolor[rgb]{0.40,0.40,0.40}{##1}}}
\expandafter\def\csname PY@tok@gp\endcsname{\let\PY@bf=\textbf\def\PY@tc##1{\textcolor[rgb]{0.00,0.00,0.50}{##1}}}
\expandafter\def\csname PY@tok@o\endcsname{\def\PY@tc##1{\textcolor[rgb]{0.40,0.40,0.40}{##1}}}
\expandafter\def\csname PY@tok@kr\endcsname{\let\PY@bf=\textbf\def\PY@tc##1{\textcolor[rgb]{0.00,0.50,0.00}{##1}}}
\expandafter\def\csname PY@tok@s\endcsname{\def\PY@tc##1{\textcolor[rgb]{0.73,0.13,0.13}{##1}}}
\expandafter\def\csname PY@tok@kp\endcsname{\def\PY@tc##1{\textcolor[rgb]{0.00,0.50,0.00}{##1}}}
\expandafter\def\csname PY@tok@w\endcsname{\def\PY@tc##1{\textcolor[rgb]{0.73,0.73,0.73}{##1}}}
\expandafter\def\csname PY@tok@kt\endcsname{\def\PY@tc##1{\textcolor[rgb]{0.69,0.00,0.25}{##1}}}
\expandafter\def\csname PY@tok@ow\endcsname{\let\PY@bf=\textbf\def\PY@tc##1{\textcolor[rgb]{0.67,0.13,1.00}{##1}}}
\expandafter\def\csname PY@tok@sb\endcsname{\def\PY@tc##1{\textcolor[rgb]{0.73,0.13,0.13}{##1}}}
\expandafter\def\csname PY@tok@k\endcsname{\let\PY@bf=\textbf\def\PY@tc##1{\textcolor[rgb]{0.00,0.50,0.00}{##1}}}
\expandafter\def\csname PY@tok@se\endcsname{\let\PY@bf=\textbf\def\PY@tc##1{\textcolor[rgb]{0.73,0.40,0.13}{##1}}}
\expandafter\def\csname PY@tok@sd\endcsname{\let\PY@it=\textit\def\PY@tc##1{\textcolor[rgb]{0.73,0.13,0.13}{##1}}}

\def\PYZbs{\char`\\}
\def\PYZus{\char`\_}
\def\PYZob{\char`\{}
\def\PYZcb{\char`\}}
\def\PYZca{\char`\^}
\def\PYZam{\char`\&}
\def\PYZlt{\char`\<}
\def\PYZgt{\char`\>}
\def\PYZsh{\char`\#}
\def\PYZpc{\char`\%}
\def\PYZdl{\char`\$}
\def\PYZhy{\char`\-}
\def\PYZsq{\char`\'}
\def\PYZdq{\char`\"}
\def\PYZti{\char`\~}
% for compatibility with earlier versions
\def\PYZat{@}
\def\PYZlb{[}
\def\PYZrb{]}
\makeatother


    % Exact colors from NB
    \definecolor{incolor}{rgb}{0.0, 0.0, 0.5}
    \definecolor{outcolor}{rgb}{0.545, 0.0, 0.0}



    
    % Prevent overflowing lines due to hard-to-break entities
    \sloppy 
    % Setup hyperref package
    \hypersetup{
      breaklinks=true,  % so long urls are correctly broken across lines
      colorlinks=true,
      urlcolor=blue,
      linkcolor=darkorange,
      citecolor=darkgreen,
      }
    % Slightly bigger margins than the latex defaults
    
     

    \begin{document}
    
    
    \maketitle
    
    

    
    \section{Functions}\label{functions}

    Most of the times, In a algorithm the statements keep repeating and it
will be a tedious job to execute the same statements again and again and
will consume a lot of memory and is not efficient. Enter Functions.

    This is the basic syntax of a function

    def funcname(arg1, arg2,\ldots{} argN):

\begin{verbatim}
''' Document String'''

statements


return <value>
\end{verbatim}

    Read the above syntax as, A function by name ``funcname'' is defined,
which accepts arguements ``arg1,arg2,\ldots{}.argN''. The function is
documented and it is `''Document String'''. The function after executing
the statements returns a ``value''.

    \begin{Verbatim}[commandchars=\\\{\}]
{\color{incolor}In [{\color{incolor}1}]:} \PY{k}{print} \PY{l+s}{\PYZdq{}}\PY{l+s}{Hey Rajath!}\PY{l+s}{\PYZdq{}}
        \PY{k}{print} \PY{l+s}{\PYZdq{}}\PY{l+s}{Rajath, How do you do?}\PY{l+s}{\PYZdq{}}
\end{Verbatim}

    \begin{Verbatim}[commandchars=\\\{\}]
Hey Rajath!
Rajath, How do you do?
    \end{Verbatim}

    Instead of writing the above two statements every single time it can be
replaced by defining a function which would do the job in just one line.

Defining a function firstfunc().

    \begin{Verbatim}[commandchars=\\\{\}]
{\color{incolor}In [{\color{incolor}2}]:} \PY{k}{def} \PY{n+nf}{firstfunc}\PY{p}{(}\PY{p}{)}\PY{p}{:}
            \PY{k}{print} \PY{l+s}{\PYZdq{}}\PY{l+s}{Hey Rajath!}\PY{l+s}{\PYZdq{}}
            \PY{k}{print} \PY{l+s}{\PYZdq{}}\PY{l+s}{Rajath, How do you do?}\PY{l+s}{\PYZdq{}}   
\end{Verbatim}

    \begin{Verbatim}[commandchars=\\\{\}]
{\color{incolor}In [{\color{incolor}3}]:} \PY{n}{firstfunc}\PY{p}{(}\PY{p}{)}
\end{Verbatim}

    \begin{Verbatim}[commandchars=\\\{\}]
Hey Rajath!
Rajath, How do you do?
    \end{Verbatim}

    \textbf{firstfunc()} every time just prints the message to a single
person. We can make our function \textbf{firstfunc()} to accept
arguements which will store the name and then prints respective to that
accepted name. To do so, add a argument within the function as shown.

    \begin{Verbatim}[commandchars=\\\{\}]
{\color{incolor}In [{\color{incolor}4}]:} \PY{k}{def} \PY{n+nf}{firstfunc}\PY{p}{(}\PY{n}{username}\PY{p}{)}\PY{p}{:}
            \PY{k}{print} \PY{l+s}{\PYZdq{}}\PY{l+s}{Hey}\PY{l+s}{\PYZdq{}}\PY{p}{,} \PY{n}{username} \PY{o}{+} \PY{l+s}{\PYZsq{}}\PY{l+s}{!}\PY{l+s}{\PYZsq{}}
            \PY{k}{print} \PY{n}{username} \PY{o}{+} \PY{l+s}{\PYZsq{}}\PY{l+s}{,}\PY{l+s}{\PYZsq{}} \PY{p}{,}\PY{l+s}{\PYZdq{}}\PY{l+s}{How do you do?}\PY{l+s}{\PYZdq{}}
\end{Verbatim}

    \begin{Verbatim}[commandchars=\\\{\}]
{\color{incolor}In [{\color{incolor}5}]:} \PY{n}{name1} \PY{o}{=} \PY{n+nb}{raw\PYZus{}input}\PY{p}{(}\PY{l+s}{\PYZsq{}}\PY{l+s}{Please enter your name : }\PY{l+s}{\PYZsq{}}\PY{p}{)}
\end{Verbatim}

    \begin{Verbatim}[commandchars=\\\{\}]
Please enter your name : Guido
    \end{Verbatim}

    The name ``Guido'' is actually stored in name1. So we pass this variable
to the function \textbf{firstfunc()} as the variable username because
that is the variable that is defined for this function. i.e name1 is
passed as username.

    \begin{Verbatim}[commandchars=\\\{\}]
{\color{incolor}In [{\color{incolor}6}]:} \PY{n}{firstfunc}\PY{p}{(}\PY{n}{name1}\PY{p}{)}
\end{Verbatim}

    \begin{Verbatim}[commandchars=\\\{\}]
Hey Guido!
Guido, How do you do?
    \end{Verbatim}

    Let us simplify this even further by defining another function
\textbf{secondfunc()} which accepts the name and stores it inside a
variable and then calls the \textbf{firstfunc()} from inside the
function itself.

    \begin{Verbatim}[commandchars=\\\{\}]
{\color{incolor}In [{\color{incolor}7}]:} \PY{k}{def} \PY{n+nf}{firstfunc}\PY{p}{(}\PY{n}{username}\PY{p}{)}\PY{p}{:}
            \PY{k}{print} \PY{l+s}{\PYZdq{}}\PY{l+s}{Hey}\PY{l+s}{\PYZdq{}}\PY{p}{,} \PY{n}{username} \PY{o}{+} \PY{l+s}{\PYZsq{}}\PY{l+s}{!}\PY{l+s}{\PYZsq{}}
            \PY{k}{print} \PY{n}{username} \PY{o}{+} \PY{l+s}{\PYZsq{}}\PY{l+s}{,}\PY{l+s}{\PYZsq{}} \PY{p}{,}\PY{l+s}{\PYZdq{}}\PY{l+s}{How do you do?}\PY{l+s}{\PYZdq{}}
        \PY{k}{def} \PY{n+nf}{secondfunc}\PY{p}{(}\PY{p}{)}\PY{p}{:}
            \PY{n}{name} \PY{o}{=} \PY{n+nb}{raw\PYZus{}input}\PY{p}{(}\PY{l+s}{\PYZdq{}}\PY{l+s}{Please enter your name : }\PY{l+s}{\PYZdq{}}\PY{p}{)}
            \PY{n}{firstfunc}\PY{p}{(}\PY{n}{name}\PY{p}{)}
\end{Verbatim}

    \begin{Verbatim}[commandchars=\\\{\}]
{\color{incolor}In [{\color{incolor}8}]:} \PY{n}{secondfunc}\PY{p}{(}\PY{p}{)}
\end{Verbatim}

    \begin{Verbatim}[commandchars=\\\{\}]
Please enter your name : karthik
Hey karthik!
karthik, How do you do?
    \end{Verbatim}

    \subsection{Return Statement}\label{return-statement}

    When the function results in some value and that value has to be stored
in a variable or needs to be sent back or returned for further operation
to the main algorithm, return statement is used.

    \begin{Verbatim}[commandchars=\\\{\}]
{\color{incolor}In [{\color{incolor}9}]:} \PY{k}{def} \PY{n+nf}{times}\PY{p}{(}\PY{n}{x}\PY{p}{,}\PY{n}{y}\PY{p}{)}\PY{p}{:}
            \PY{n}{z} \PY{o}{=} \PY{n}{x}\PY{o}{*}\PY{n}{y}
            \PY{k}{return} \PY{n}{z}
\end{Verbatim}

    The above defined \textbf{times( )} function accepts two arguements and
return the variable z which contains the result of the product of the
two arguements

    \begin{Verbatim}[commandchars=\\\{\}]
{\color{incolor}In [{\color{incolor}10}]:} \PY{n}{c} \PY{o}{=} \PY{n}{times}\PY{p}{(}\PY{l+m+mi}{4}\PY{p}{,}\PY{l+m+mi}{5}\PY{p}{)}
         \PY{k}{print} \PY{n}{c}
\end{Verbatim}

    \begin{Verbatim}[commandchars=\\\{\}]
20
    \end{Verbatim}

    The z value is stored in variable c and can be used for further
operations.

    Instead of declaring another variable the entire statement itself can be
used in the return statement as shown.

    \begin{Verbatim}[commandchars=\\\{\}]
{\color{incolor}In [{\color{incolor}11}]:} \PY{k}{def} \PY{n+nf}{times}\PY{p}{(}\PY{n}{x}\PY{p}{,}\PY{n}{y}\PY{p}{)}\PY{p}{:}
             \PY{l+s+sd}{\PYZsq{}\PYZsq{}\PYZsq{}This multiplies the two input arguments\PYZsq{}\PYZsq{}\PYZsq{}}
             \PY{k}{return} \PY{n}{x}\PY{o}{*}\PY{n}{y}
\end{Verbatim}

    \begin{Verbatim}[commandchars=\\\{\}]
{\color{incolor}In [{\color{incolor}12}]:} \PY{n}{c} \PY{o}{=} \PY{n}{times}\PY{p}{(}\PY{l+m+mi}{4}\PY{p}{,}\PY{l+m+mi}{5}\PY{p}{)}
         \PY{k}{print} \PY{n}{c}
\end{Verbatim}

    \begin{Verbatim}[commandchars=\\\{\}]
20
    \end{Verbatim}

    Since the \textbf{times( )} is now defined, we can document it as shown
above. This document is returned whenever \textbf{times( )} function is
called under \textbf{help( )} function.

    \begin{Verbatim}[commandchars=\\\{\}]
{\color{incolor}In [{\color{incolor}13}]:} \PY{n}{help}\PY{p}{(}\PY{n}{times}\PY{p}{)}
\end{Verbatim}

    \begin{Verbatim}[commandchars=\\\{\}]
Help on function times in module \_\_main\_\_:

times(x, y)
    This multiplies the two input arguments
    \end{Verbatim}

    Multiple variable can also be returned, But keep in mind the order.

    \begin{Verbatim}[commandchars=\\\{\}]
{\color{incolor}In [{\color{incolor}14}]:} \PY{n}{eglist} \PY{o}{=} \PY{p}{[}\PY{l+m+mi}{10}\PY{p}{,}\PY{l+m+mi}{50}\PY{p}{,}\PY{l+m+mi}{30}\PY{p}{,}\PY{l+m+mi}{12}\PY{p}{,}\PY{l+m+mi}{6}\PY{p}{,}\PY{l+m+mi}{8}\PY{p}{,}\PY{l+m+mi}{100}\PY{p}{]}
\end{Verbatim}

    \begin{Verbatim}[commandchars=\\\{\}]
{\color{incolor}In [{\color{incolor}15}]:} \PY{k}{def} \PY{n+nf}{egfunc}\PY{p}{(}\PY{n}{eglist}\PY{p}{)}\PY{p}{:}
             \PY{n}{highest} \PY{o}{=} \PY{n+nb}{max}\PY{p}{(}\PY{n}{eglist}\PY{p}{)}
             \PY{n}{lowest} \PY{o}{=} \PY{n+nb}{min}\PY{p}{(}\PY{n}{eglist}\PY{p}{)}
             \PY{n}{first} \PY{o}{=} \PY{n}{eglist}\PY{p}{[}\PY{l+m+mi}{0}\PY{p}{]}
             \PY{n}{last} \PY{o}{=} \PY{n}{eglist}\PY{p}{[}\PY{o}{\PYZhy{}}\PY{l+m+mi}{1}\PY{p}{]}
             \PY{k}{return} \PY{n}{highest}\PY{p}{,}\PY{n}{lowest}\PY{p}{,}\PY{n}{first}\PY{p}{,}\PY{n}{last}
\end{Verbatim}

    If the function is just called without any variable for it to be
assigned to, the result is returned inside a tuple. But if the variables
are mentioned then the result is assigned to the variable in a
particular order which is declared in the return statement.

    \begin{Verbatim}[commandchars=\\\{\}]
{\color{incolor}In [{\color{incolor}16}]:} \PY{n}{egfunc}\PY{p}{(}\PY{n}{eglist}\PY{p}{)}
\end{Verbatim}

            \begin{Verbatim}[commandchars=\\\{\}]
{\color{outcolor}Out[{\color{outcolor}16}]:} (100, 6, 10, 100)
\end{Verbatim}
        
    \begin{Verbatim}[commandchars=\\\{\}]
{\color{incolor}In [{\color{incolor}17}]:} \PY{n}{a}\PY{p}{,}\PY{n}{b}\PY{p}{,}\PY{n}{c}\PY{p}{,}\PY{n}{d} \PY{o}{=} \PY{n}{egfunc}\PY{p}{(}\PY{n}{eglist}\PY{p}{)}
         \PY{k}{print} \PY{l+s}{\PYZsq{}}\PY{l+s}{ a =}\PY{l+s}{\PYZsq{}}\PY{p}{,}\PY{n}{a}\PY{p}{,}\PY{l+s}{\PYZsq{}}\PY{l+s+se}{\PYZbs{}n}\PY{l+s}{ b =}\PY{l+s}{\PYZsq{}}\PY{p}{,}\PY{n}{b}\PY{p}{,}\PY{l+s}{\PYZsq{}}\PY{l+s+se}{\PYZbs{}n}\PY{l+s}{ c =}\PY{l+s}{\PYZsq{}}\PY{p}{,}\PY{n}{c}\PY{p}{,}\PY{l+s}{\PYZsq{}}\PY{l+s+se}{\PYZbs{}n}\PY{l+s}{ d =}\PY{l+s}{\PYZsq{}}\PY{p}{,}\PY{n}{d}
\end{Verbatim}

    \begin{Verbatim}[commandchars=\\\{\}]
a = 100 
 b = 6 
 c = 10 
 d = 100
    \end{Verbatim}

    \subsection{Implicit arguments}\label{implicit-arguments}

    When an argument of a function is common in majority of the cases or it
is ``implicit'' this concept is used.

    \begin{Verbatim}[commandchars=\\\{\}]
{\color{incolor}In [{\color{incolor}18}]:} \PY{k}{def} \PY{n+nf}{implicitadd}\PY{p}{(}\PY{n}{x}\PY{p}{,}\PY{n}{y}\PY{o}{=}\PY{l+m+mi}{3}\PY{p}{)}\PY{p}{:}
             \PY{k}{return} \PY{n}{x}\PY{o}{+}\PY{n}{y}
\end{Verbatim}

    \textbf{implicitadd( )} is a function accepts two arguments but most of
the times the first argument needs to be added just by 3. Hence the
second argument is assigned the value 3. Here the second argument is
implicit.

    Now if the second argument is not defined when calling the
\textbf{implicitadd( )} function then it considered as 3.

    \begin{Verbatim}[commandchars=\\\{\}]
{\color{incolor}In [{\color{incolor}19}]:} \PY{n}{implicitadd}\PY{p}{(}\PY{l+m+mi}{4}\PY{p}{)}
\end{Verbatim}

            \begin{Verbatim}[commandchars=\\\{\}]
{\color{outcolor}Out[{\color{outcolor}19}]:} 7
\end{Verbatim}
        
    But if the second argument is specified then this value overrides the
implicit value assigned to the argument

    \begin{Verbatim}[commandchars=\\\{\}]
{\color{incolor}In [{\color{incolor}20}]:} \PY{n}{implicitadd}\PY{p}{(}\PY{l+m+mi}{4}\PY{p}{,}\PY{l+m+mi}{4}\PY{p}{)}
\end{Verbatim}

            \begin{Verbatim}[commandchars=\\\{\}]
{\color{outcolor}Out[{\color{outcolor}20}]:} 8
\end{Verbatim}
        
    \subsection{Any number of arguments}\label{any-number-of-arguments}

    If the number of arguments that is to be accepted by a function is not
known then a asterisk symbol is used before the argument.

    \begin{Verbatim}[commandchars=\\\{\}]
{\color{incolor}In [{\color{incolor}21}]:} \PY{k}{def} \PY{n+nf}{add\PYZus{}n}\PY{p}{(}\PY{o}{*}\PY{n}{args}\PY{p}{)}\PY{p}{:}
             \PY{n}{res} \PY{o}{=} \PY{l+m+mi}{0}
             \PY{n}{reslist} \PY{o}{=} \PY{p}{[}\PY{p}{]}
             \PY{k}{for} \PY{n}{i} \PY{o+ow}{in} \PY{n}{args}\PY{p}{:}
                 \PY{n}{reslist}\PY{o}{.}\PY{n}{append}\PY{p}{(}\PY{n}{i}\PY{p}{)}
             \PY{k}{print} \PY{n}{reslist}
             \PY{k}{return} \PY{n+nb}{sum}\PY{p}{(}\PY{n}{reslist}\PY{p}{)}
\end{Verbatim}

    The above function accepts any number of arguments, defines a list and
appends all the arguments into that list and return the sum of all the
arguments.

    \begin{Verbatim}[commandchars=\\\{\}]
{\color{incolor}In [{\color{incolor}22}]:} \PY{n}{add\PYZus{}n}\PY{p}{(}\PY{l+m+mi}{1}\PY{p}{,}\PY{l+m+mi}{2}\PY{p}{,}\PY{l+m+mi}{3}\PY{p}{,}\PY{l+m+mi}{4}\PY{p}{,}\PY{l+m+mi}{5}\PY{p}{)}
\end{Verbatim}

    \begin{Verbatim}[commandchars=\\\{\}]
[1, 2, 3, 4, 5]
    \end{Verbatim}

            \begin{Verbatim}[commandchars=\\\{\}]
{\color{outcolor}Out[{\color{outcolor}22}]:} 15
\end{Verbatim}
        
    \begin{Verbatim}[commandchars=\\\{\}]
{\color{incolor}In [{\color{incolor}23}]:} \PY{n}{add\PYZus{}n}\PY{p}{(}\PY{l+m+mi}{1}\PY{p}{,}\PY{l+m+mi}{2}\PY{p}{,}\PY{l+m+mi}{3}\PY{p}{)}
\end{Verbatim}

    \begin{Verbatim}[commandchars=\\\{\}]
[1, 2, 3]
    \end{Verbatim}

            \begin{Verbatim}[commandchars=\\\{\}]
{\color{outcolor}Out[{\color{outcolor}23}]:} 6
\end{Verbatim}
        
    \subsection{Global and Local
Variables}\label{global-and-local-variables}

    Whatever variable is declared inside a function is local variable and
outside the function in global variable.

    \begin{Verbatim}[commandchars=\\\{\}]
{\color{incolor}In [{\color{incolor}24}]:} \PY{n}{eg1} \PY{o}{=} \PY{p}{[}\PY{l+m+mi}{1}\PY{p}{,}\PY{l+m+mi}{2}\PY{p}{,}\PY{l+m+mi}{3}\PY{p}{,}\PY{l+m+mi}{4}\PY{p}{,}\PY{l+m+mi}{5}\PY{p}{]}
\end{Verbatim}

    In the below function we are appending a element to the declared list
inside the function. eg2 variable declared inside the function is a
local variable.

    \begin{Verbatim}[commandchars=\\\{\}]
{\color{incolor}In [{\color{incolor}25}]:} \PY{k}{def} \PY{n+nf}{egfunc1}\PY{p}{(}\PY{p}{)}\PY{p}{:}
             \PY{k}{def} \PY{n+nf}{thirdfunc}\PY{p}{(}\PY{n}{arg1}\PY{p}{)}\PY{p}{:}
                 \PY{n}{eg2} \PY{o}{=} \PY{n}{arg1}\PY{p}{[}\PY{p}{:}\PY{p}{]}
                 \PY{n}{eg2}\PY{o}{.}\PY{n}{append}\PY{p}{(}\PY{l+m+mi}{6}\PY{p}{)}
                 \PY{k}{print} \PY{l+s}{\PYZdq{}}\PY{l+s}{This is happening inside the function :}\PY{l+s}{\PYZdq{}}\PY{p}{,} \PY{n}{eg2} 
             \PY{k}{print} \PY{l+s}{\PYZdq{}}\PY{l+s}{This is happening before the function is called : }\PY{l+s}{\PYZdq{}}\PY{p}{,} \PY{n}{eg1}
             \PY{n}{thirdfunc}\PY{p}{(}\PY{n}{eg1}\PY{p}{)}
             \PY{k}{print} \PY{l+s}{\PYZdq{}}\PY{l+s}{This is happening outside the function :}\PY{l+s}{\PYZdq{}}\PY{p}{,} \PY{n}{eg1}   
             \PY{k}{print} \PY{l+s}{\PYZdq{}}\PY{l+s}{Accessing a variable declared inside the function from outside :}\PY{l+s}{\PYZdq{}} \PY{p}{,} \PY{n}{eg2}
\end{Verbatim}

    \begin{Verbatim}[commandchars=\\\{\}]
{\color{incolor}In [{\color{incolor}26}]:} \PY{n}{egfunc1}\PY{p}{(}\PY{p}{)}
\end{Verbatim}

    \begin{Verbatim}[commandchars=\\\{\}]
This is happening before the function is called :  [1, 2, 3, 4, 5]
This is happening inside the function : [1, 2, 3, 4, 5, 6]
This is happening outside the function : [1, 2, 3, 4, 5]
Accessing a variable declared inside the function from outside :
    \end{Verbatim}

    \begin{Verbatim}[commandchars=\\\{\}]

        ---------------------------------------------------------------------------

        NameError                                 Traceback (most recent call last)

        <ipython-input-26-949117e1ddc5> in <module>()
    ----> 1 egfunc1()
    

        <ipython-input-25-0da329480da9> in egfunc1()
          7     thirdfunc(eg1)
          8     print "This is happening outside the function :", eg1
    ----> 9     print "Accessing a variable declared inside the function from outside :" , eg2
    

        NameError: global name 'eg2' is not defined

    \end{Verbatim}

    If a \textbf{global} variable is defined as shown in the example below
then that variable can be called from anywhere.

    \begin{Verbatim}[commandchars=\\\{\}]
{\color{incolor}In [{\color{incolor}27}]:} \PY{n}{eg3} \PY{o}{=} \PY{p}{[}\PY{l+m+mi}{1}\PY{p}{,}\PY{l+m+mi}{2}\PY{p}{,}\PY{l+m+mi}{3}\PY{p}{,}\PY{l+m+mi}{4}\PY{p}{,}\PY{l+m+mi}{5}\PY{p}{]}
\end{Verbatim}

    \begin{Verbatim}[commandchars=\\\{\}]

    \end{Verbatim}

    \begin{Verbatim}[commandchars=\\\{\}]
{\color{incolor}In [{\color{incolor}28}]:} \PY{k}{def} \PY{n+nf}{egfunc1}\PY{p}{(}\PY{p}{)}\PY{p}{:}
             \PY{k}{def} \PY{n+nf}{thirdfunc}\PY{p}{(}\PY{n}{arg1}\PY{p}{)}\PY{p}{:}
                 \PY{k}{global} \PY{n}{eg2}
                 \PY{n}{eg2} \PY{o}{=} \PY{n}{arg1}\PY{p}{[}\PY{p}{:}\PY{p}{]}
                 \PY{n}{eg2}\PY{o}{.}\PY{n}{append}\PY{p}{(}\PY{l+m+mi}{6}\PY{p}{)}
                 \PY{k}{print} \PY{l+s}{\PYZdq{}}\PY{l+s}{This is happening inside the function :}\PY{l+s}{\PYZdq{}}\PY{p}{,} \PY{n}{eg2} 
             \PY{k}{print} \PY{l+s}{\PYZdq{}}\PY{l+s}{This is happening before the function is called : }\PY{l+s}{\PYZdq{}}\PY{p}{,} \PY{n}{eg1}
             \PY{n}{thirdfunc}\PY{p}{(}\PY{n}{eg1}\PY{p}{)}
             \PY{k}{print} \PY{l+s}{\PYZdq{}}\PY{l+s}{This is happening outside the function :}\PY{l+s}{\PYZdq{}}\PY{p}{,} \PY{n}{eg1}   
             \PY{k}{print} \PY{l+s}{\PYZdq{}}\PY{l+s}{Accessing a variable declared inside the function from outside :}\PY{l+s}{\PYZdq{}} \PY{p}{,} \PY{n}{eg2}
\end{Verbatim}

    \begin{Verbatim}[commandchars=\\\{\}]
{\color{incolor}In [{\color{incolor}29}]:} \PY{n}{egfunc1}\PY{p}{(}\PY{p}{)}
\end{Verbatim}

    \begin{Verbatim}[commandchars=\\\{\}]
This is happening before the function is called :  [1, 2, 3, 4, 5]
This is happening inside the function : [1, 2, 3, 4, 5, 6]
This is happening outside the function : [1, 2, 3, 4, 5]
Accessing a variable declared inside the function from outside : [1, 2, 3, 4, 5, 6]
    \end{Verbatim}

    \subsection{Lambda Functions}\label{lambda-functions}

    These are small functions which are not defined with any name and carry
a single expression whose result is returned. Lambda functions comes
very handy when operating with lists. These function are defined by the
keyword \textbf{lambda} followed by the variables, a colon and the
respective expression.

    \begin{Verbatim}[commandchars=\\\{\}]
{\color{incolor}In [{\color{incolor}30}]:} \PY{n}{z} \PY{o}{=} \PY{k}{lambda} \PY{n}{x}\PY{p}{:} \PY{n}{x} \PY{o}{*} \PY{n}{x}
\end{Verbatim}

    \begin{Verbatim}[commandchars=\\\{\}]
{\color{incolor}In [{\color{incolor}31}]:} \PY{n}{z}\PY{p}{(}\PY{l+m+mi}{8}\PY{p}{)}
\end{Verbatim}

            \begin{Verbatim}[commandchars=\\\{\}]
{\color{outcolor}Out[{\color{outcolor}31}]:} 64
\end{Verbatim}
        
    \subsubsection{map}\label{map}

    \textbf{map( )} function basically executes the function that is defined
to each of the list's element separately.

    \begin{Verbatim}[commandchars=\\\{\}]
{\color{incolor}In [{\color{incolor}32}]:} \PY{n}{list1} \PY{o}{=} \PY{p}{[}\PY{l+m+mi}{1}\PY{p}{,}\PY{l+m+mi}{2}\PY{p}{,}\PY{l+m+mi}{3}\PY{p}{,}\PY{l+m+mi}{4}\PY{p}{,}\PY{l+m+mi}{5}\PY{p}{,}\PY{l+m+mi}{6}\PY{p}{,}\PY{l+m+mi}{7}\PY{p}{,}\PY{l+m+mi}{8}\PY{p}{,}\PY{l+m+mi}{9}\PY{p}{]}
\end{Verbatim}

    \begin{Verbatim}[commandchars=\\\{\}]
{\color{incolor}In [{\color{incolor}33}]:} \PY{n}{eg} \PY{o}{=} \PY{n+nb}{map}\PY{p}{(}\PY{k}{lambda} \PY{n}{x}\PY{p}{:}\PY{n}{x}\PY{o}{+}\PY{l+m+mi}{2}\PY{p}{,} \PY{n}{list1}\PY{p}{)}
         \PY{k}{print} \PY{n}{eg}
\end{Verbatim}

    \begin{Verbatim}[commandchars=\\\{\}]
[3, 4, 5, 6, 7, 8, 9, 10, 11]
    \end{Verbatim}

    You can also add two lists.

    \begin{Verbatim}[commandchars=\\\{\}]
{\color{incolor}In [{\color{incolor}34}]:} \PY{n}{list2} \PY{o}{=} \PY{p}{[}\PY{l+m+mi}{9}\PY{p}{,}\PY{l+m+mi}{8}\PY{p}{,}\PY{l+m+mi}{7}\PY{p}{,}\PY{l+m+mi}{6}\PY{p}{,}\PY{l+m+mi}{5}\PY{p}{,}\PY{l+m+mi}{4}\PY{p}{,}\PY{l+m+mi}{3}\PY{p}{,}\PY{l+m+mi}{2}\PY{p}{,}\PY{l+m+mi}{1}\PY{p}{]}
\end{Verbatim}

    \begin{Verbatim}[commandchars=\\\{\}]
{\color{incolor}In [{\color{incolor}35}]:} \PY{n}{eg2} \PY{o}{=} \PY{n+nb}{map}\PY{p}{(}\PY{k}{lambda} \PY{n}{x}\PY{p}{,}\PY{n}{y}\PY{p}{:}\PY{n}{x}\PY{o}{+}\PY{n}{y}\PY{p}{,} \PY{n}{list1}\PY{p}{,}\PY{n}{list2}\PY{p}{)}
         \PY{k}{print} \PY{n}{eg2}
\end{Verbatim}

    \begin{Verbatim}[commandchars=\\\{\}]
[10, 10, 10, 10, 10, 10, 10, 10, 10]
    \end{Verbatim}

    Not only lambda function but also other built in functions can also be
used.

    \begin{Verbatim}[commandchars=\\\{\}]
{\color{incolor}In [{\color{incolor}36}]:} \PY{n}{eg3} \PY{o}{=} \PY{n+nb}{map}\PY{p}{(}\PY{n+nb}{str}\PY{p}{,}\PY{n}{eg2}\PY{p}{)}
         \PY{k}{print} \PY{n}{eg3}
\end{Verbatim}

    \begin{Verbatim}[commandchars=\\\{\}]
['10', '10', '10', '10', '10', '10', '10', '10', '10']
    \end{Verbatim}

    \subsubsection{filter}\label{filter}

    \textbf{filter( )} function is used to filter out the values in a list.
Note that \textbf{filter()} function returns the result in a new list.

    \begin{Verbatim}[commandchars=\\\{\}]
{\color{incolor}In [{\color{incolor}37}]:} \PY{n}{list1} \PY{o}{=} \PY{p}{[}\PY{l+m+mi}{1}\PY{p}{,}\PY{l+m+mi}{2}\PY{p}{,}\PY{l+m+mi}{3}\PY{p}{,}\PY{l+m+mi}{4}\PY{p}{,}\PY{l+m+mi}{5}\PY{p}{,}\PY{l+m+mi}{6}\PY{p}{,}\PY{l+m+mi}{7}\PY{p}{,}\PY{l+m+mi}{8}\PY{p}{,}\PY{l+m+mi}{9}\PY{p}{]}
\end{Verbatim}

    To get the elements which are less than 5,

    \begin{Verbatim}[commandchars=\\\{\}]
{\color{incolor}In [{\color{incolor}38}]:} \PY{n+nb}{filter}\PY{p}{(}\PY{k}{lambda} \PY{n}{x}\PY{p}{:}\PY{n}{x}\PY{o}{\PYZlt{}}\PY{l+m+mi}{5}\PY{p}{,}\PY{n}{list1}\PY{p}{)}
\end{Verbatim}

            \begin{Verbatim}[commandchars=\\\{\}]
{\color{outcolor}Out[{\color{outcolor}38}]:} [1, 2, 3, 4]
\end{Verbatim}
        
    Notice what happens when \textbf{map()} is used.

    \begin{Verbatim}[commandchars=\\\{\}]
{\color{incolor}In [{\color{incolor}39}]:} \PY{n+nb}{map}\PY{p}{(}\PY{k}{lambda} \PY{n}{x}\PY{p}{:}\PY{n}{x}\PY{o}{\PYZlt{}}\PY{l+m+mi}{5}\PY{p}{,} \PY{n}{list1}\PY{p}{)}
\end{Verbatim}

            \begin{Verbatim}[commandchars=\\\{\}]
{\color{outcolor}Out[{\color{outcolor}39}]:} [True, True, True, True, False, False, False, False, False]
\end{Verbatim}
        
    We can conclude that, whatever is returned true in \textbf{map( )}
function that particular element is returned when \textbf{filter( )}
function is used.

    \begin{Verbatim}[commandchars=\\\{\}]
{\color{incolor}In [{\color{incolor}40}]:} \PY{n+nb}{filter}\PY{p}{(}\PY{k}{lambda} \PY{n}{x}\PY{p}{:}\PY{n}{x}\PY{o}{\PYZpc{}}\PY{k}{4}==0,list1)
\end{Verbatim}

            \begin{Verbatim}[commandchars=\\\{\}]
{\color{outcolor}Out[{\color{outcolor}40}]:} [4, 8]
\end{Verbatim}
        

    % Add a bibliography block to the postdoc
    
  \newpage
  
  
% Default to the notebook output style

    


% Inherit from the specified cell style.




    
    
    
    \definecolor{orange}{cmyk}{0,0.4,0.8,0.2}
    \definecolor{darkorange}{rgb}{.71,0.21,0.01}
    \definecolor{darkgreen}{rgb}{.12,.54,.11}
    \definecolor{myteal}{rgb}{.26, .44, .56}
    \definecolor{gray}{gray}{0.45}
    \definecolor{lightgray}{gray}{.95}
    \definecolor{mediumgray}{gray}{.8}
    \definecolor{inputbackground}{rgb}{.95, .95, .85}
    \definecolor{outputbackground}{rgb}{.95, .95, .95}
    \definecolor{traceback}{rgb}{1, .95, .95}
    % ansi colors
    \definecolor{red}{rgb}{.6,0,0}
    \definecolor{green}{rgb}{0,.65,0}
    \definecolor{brown}{rgb}{0.6,0.6,0}
    \definecolor{blue}{rgb}{0,.145,.698}
    \definecolor{purple}{rgb}{.698,.145,.698}
    \definecolor{cyan}{rgb}{0,.698,.698}
    \definecolor{lightgray}{gray}{0.5}
    
    % bright ansi colors
    \definecolor{darkgray}{gray}{0.25}
    \definecolor{lightred}{rgb}{1.0,0.39,0.28}
    \definecolor{lightgreen}{rgb}{0.48,0.99,0.0}
    \definecolor{lightblue}{rgb}{0.53,0.81,0.92}
    \definecolor{lightpurple}{rgb}{0.87,0.63,0.87}
    \definecolor{lightcyan}{rgb}{0.5,1.0,0.83}
    
    % commands and environments needed by pandoc snippets
    % extracted from the output of `pandoc -s`
    \DefineVerbatimEnvironment{Highlighting}{Verbatim}{commandchars=\\\{\}}
    % Add ',fontsize=\small' for more characters per line
    \newenvironment{Shaded}{}{}
    \newcommand{\KeywordTok}[1]{\textcolor[rgb]{0.00,0.44,0.13}{\textbf{{#1}}}}
    \newcommand{\DataTypeTok}[1]{\textcolor[rgb]{0.56,0.13,0.00}{{#1}}}
    \newcommand{\DecValTok}[1]{\textcolor[rgb]{0.25,0.63,0.44}{{#1}}}
    \newcommand{\BaseNTok}[1]{\textcolor[rgb]{0.25,0.63,0.44}{{#1}}}
    \newcommand{\FloatTok}[1]{\textcolor[rgb]{0.25,0.63,0.44}{{#1}}}
    \newcommand{\CharTok}[1]{\textcolor[rgb]{0.25,0.44,0.63}{{#1}}}
    \newcommand{\StringTok}[1]{\textcolor[rgb]{0.25,0.44,0.63}{{#1}}}
    \newcommand{\CommentTok}[1]{\textcolor[rgb]{0.38,0.63,0.69}{\textit{{#1}}}}
    \newcommand{\OtherTok}[1]{\textcolor[rgb]{0.00,0.44,0.13}{{#1}}}
    \newcommand{\AlertTok}[1]{\textcolor[rgb]{1.00,0.00,0.00}{\textbf{{#1}}}}
    \newcommand{\FunctionTok}[1]{\textcolor[rgb]{0.02,0.16,0.49}{{#1}}}
    \newcommand{\RegionMarkerTok}[1]{{#1}}
    \newcommand{\ErrorTok}[1]{\textcolor[rgb]{1.00,0.00,0.00}{\textbf{{#1}}}}
    \newcommand{\NormalTok}[1]{{#1}}
    
    % Define a nice break command that doesn't care if a line doesn't already
    % exist.
    \def\br{\hspace*{\fill} \\* }
    % Math Jax compatability definitions
    \def\gt{>}
    \def\lt{<}
    % Document parameters
    \title{}
    
    
    

    % Pygments definitions
    
\makeatletter
\def\PY@reset{\let\PY@it=\relax \let\PY@bf=\relax%
    \let\PY@ul=\relax \let\PY@tc=\relax%
    \let\PY@bc=\relax \let\PY@ff=\relax}
\def\PY@tok#1{\csname PY@tok@#1\endcsname}
\def\PY@toks#1+{\ifx\relax#1\empty\else%
    \PY@tok{#1}\expandafter\PY@toks\fi}
\def\PY@do#1{\PY@bc{\PY@tc{\PY@ul{%
    \PY@it{\PY@bf{\PY@ff{#1}}}}}}}
\def\PY#1#2{\PY@reset\PY@toks#1+\relax+\PY@do{#2}}

\expandafter\def\csname PY@tok@gd\endcsname{\def\PY@tc##1{\textcolor[rgb]{0.63,0.00,0.00}{##1}}}
\expandafter\def\csname PY@tok@gu\endcsname{\let\PY@bf=\textbf\def\PY@tc##1{\textcolor[rgb]{0.50,0.00,0.50}{##1}}}
\expandafter\def\csname PY@tok@gt\endcsname{\def\PY@tc##1{\textcolor[rgb]{0.00,0.27,0.87}{##1}}}
\expandafter\def\csname PY@tok@gs\endcsname{\let\PY@bf=\textbf}
\expandafter\def\csname PY@tok@gr\endcsname{\def\PY@tc##1{\textcolor[rgb]{1.00,0.00,0.00}{##1}}}
\expandafter\def\csname PY@tok@cm\endcsname{\let\PY@it=\textit\def\PY@tc##1{\textcolor[rgb]{0.25,0.50,0.50}{##1}}}
\expandafter\def\csname PY@tok@vg\endcsname{\def\PY@tc##1{\textcolor[rgb]{0.10,0.09,0.49}{##1}}}
\expandafter\def\csname PY@tok@m\endcsname{\def\PY@tc##1{\textcolor[rgb]{0.40,0.40,0.40}{##1}}}
\expandafter\def\csname PY@tok@mh\endcsname{\def\PY@tc##1{\textcolor[rgb]{0.40,0.40,0.40}{##1}}}
\expandafter\def\csname PY@tok@go\endcsname{\def\PY@tc##1{\textcolor[rgb]{0.53,0.53,0.53}{##1}}}
\expandafter\def\csname PY@tok@ge\endcsname{\let\PY@it=\textit}
\expandafter\def\csname PY@tok@vc\endcsname{\def\PY@tc##1{\textcolor[rgb]{0.10,0.09,0.49}{##1}}}
\expandafter\def\csname PY@tok@il\endcsname{\def\PY@tc##1{\textcolor[rgb]{0.40,0.40,0.40}{##1}}}
\expandafter\def\csname PY@tok@cs\endcsname{\let\PY@it=\textit\def\PY@tc##1{\textcolor[rgb]{0.25,0.50,0.50}{##1}}}
\expandafter\def\csname PY@tok@cp\endcsname{\def\PY@tc##1{\textcolor[rgb]{0.74,0.48,0.00}{##1}}}
\expandafter\def\csname PY@tok@gi\endcsname{\def\PY@tc##1{\textcolor[rgb]{0.00,0.63,0.00}{##1}}}
\expandafter\def\csname PY@tok@gh\endcsname{\let\PY@bf=\textbf\def\PY@tc##1{\textcolor[rgb]{0.00,0.00,0.50}{##1}}}
\expandafter\def\csname PY@tok@ni\endcsname{\let\PY@bf=\textbf\def\PY@tc##1{\textcolor[rgb]{0.60,0.60,0.60}{##1}}}
\expandafter\def\csname PY@tok@nl\endcsname{\def\PY@tc##1{\textcolor[rgb]{0.63,0.63,0.00}{##1}}}
\expandafter\def\csname PY@tok@nn\endcsname{\let\PY@bf=\textbf\def\PY@tc##1{\textcolor[rgb]{0.00,0.00,1.00}{##1}}}
\expandafter\def\csname PY@tok@no\endcsname{\def\PY@tc##1{\textcolor[rgb]{0.53,0.00,0.00}{##1}}}
\expandafter\def\csname PY@tok@na\endcsname{\def\PY@tc##1{\textcolor[rgb]{0.49,0.56,0.16}{##1}}}
\expandafter\def\csname PY@tok@nb\endcsname{\def\PY@tc##1{\textcolor[rgb]{0.00,0.50,0.00}{##1}}}
\expandafter\def\csname PY@tok@nc\endcsname{\let\PY@bf=\textbf\def\PY@tc##1{\textcolor[rgb]{0.00,0.00,1.00}{##1}}}
\expandafter\def\csname PY@tok@nd\endcsname{\def\PY@tc##1{\textcolor[rgb]{0.67,0.13,1.00}{##1}}}
\expandafter\def\csname PY@tok@ne\endcsname{\let\PY@bf=\textbf\def\PY@tc##1{\textcolor[rgb]{0.82,0.25,0.23}{##1}}}
\expandafter\def\csname PY@tok@nf\endcsname{\def\PY@tc##1{\textcolor[rgb]{0.00,0.00,1.00}{##1}}}
\expandafter\def\csname PY@tok@si\endcsname{\let\PY@bf=\textbf\def\PY@tc##1{\textcolor[rgb]{0.73,0.40,0.53}{##1}}}
\expandafter\def\csname PY@tok@s2\endcsname{\def\PY@tc##1{\textcolor[rgb]{0.73,0.13,0.13}{##1}}}
\expandafter\def\csname PY@tok@vi\endcsname{\def\PY@tc##1{\textcolor[rgb]{0.10,0.09,0.49}{##1}}}
\expandafter\def\csname PY@tok@nt\endcsname{\let\PY@bf=\textbf\def\PY@tc##1{\textcolor[rgb]{0.00,0.50,0.00}{##1}}}
\expandafter\def\csname PY@tok@nv\endcsname{\def\PY@tc##1{\textcolor[rgb]{0.10,0.09,0.49}{##1}}}
\expandafter\def\csname PY@tok@s1\endcsname{\def\PY@tc##1{\textcolor[rgb]{0.73,0.13,0.13}{##1}}}
\expandafter\def\csname PY@tok@kd\endcsname{\let\PY@bf=\textbf\def\PY@tc##1{\textcolor[rgb]{0.00,0.50,0.00}{##1}}}
\expandafter\def\csname PY@tok@sh\endcsname{\def\PY@tc##1{\textcolor[rgb]{0.73,0.13,0.13}{##1}}}
\expandafter\def\csname PY@tok@sc\endcsname{\def\PY@tc##1{\textcolor[rgb]{0.73,0.13,0.13}{##1}}}
\expandafter\def\csname PY@tok@sx\endcsname{\def\PY@tc##1{\textcolor[rgb]{0.00,0.50,0.00}{##1}}}
\expandafter\def\csname PY@tok@bp\endcsname{\def\PY@tc##1{\textcolor[rgb]{0.00,0.50,0.00}{##1}}}
\expandafter\def\csname PY@tok@c1\endcsname{\let\PY@it=\textit\def\PY@tc##1{\textcolor[rgb]{0.25,0.50,0.50}{##1}}}
\expandafter\def\csname PY@tok@kc\endcsname{\let\PY@bf=\textbf\def\PY@tc##1{\textcolor[rgb]{0.00,0.50,0.00}{##1}}}
\expandafter\def\csname PY@tok@c\endcsname{\let\PY@it=\textit\def\PY@tc##1{\textcolor[rgb]{0.25,0.50,0.50}{##1}}}
\expandafter\def\csname PY@tok@mf\endcsname{\def\PY@tc##1{\textcolor[rgb]{0.40,0.40,0.40}{##1}}}
\expandafter\def\csname PY@tok@err\endcsname{\def\PY@bc##1{\setlength{\fboxsep}{0pt}\fcolorbox[rgb]{1.00,0.00,0.00}{1,1,1}{\strut ##1}}}
\expandafter\def\csname PY@tok@mb\endcsname{\def\PY@tc##1{\textcolor[rgb]{0.40,0.40,0.40}{##1}}}
\expandafter\def\csname PY@tok@ss\endcsname{\def\PY@tc##1{\textcolor[rgb]{0.10,0.09,0.49}{##1}}}
\expandafter\def\csname PY@tok@sr\endcsname{\def\PY@tc##1{\textcolor[rgb]{0.73,0.40,0.53}{##1}}}
\expandafter\def\csname PY@tok@mo\endcsname{\def\PY@tc##1{\textcolor[rgb]{0.40,0.40,0.40}{##1}}}
\expandafter\def\csname PY@tok@kn\endcsname{\let\PY@bf=\textbf\def\PY@tc##1{\textcolor[rgb]{0.00,0.50,0.00}{##1}}}
\expandafter\def\csname PY@tok@mi\endcsname{\def\PY@tc##1{\textcolor[rgb]{0.40,0.40,0.40}{##1}}}
\expandafter\def\csname PY@tok@gp\endcsname{\let\PY@bf=\textbf\def\PY@tc##1{\textcolor[rgb]{0.00,0.00,0.50}{##1}}}
\expandafter\def\csname PY@tok@o\endcsname{\def\PY@tc##1{\textcolor[rgb]{0.40,0.40,0.40}{##1}}}
\expandafter\def\csname PY@tok@kr\endcsname{\let\PY@bf=\textbf\def\PY@tc##1{\textcolor[rgb]{0.00,0.50,0.00}{##1}}}
\expandafter\def\csname PY@tok@s\endcsname{\def\PY@tc##1{\textcolor[rgb]{0.73,0.13,0.13}{##1}}}
\expandafter\def\csname PY@tok@kp\endcsname{\def\PY@tc##1{\textcolor[rgb]{0.00,0.50,0.00}{##1}}}
\expandafter\def\csname PY@tok@w\endcsname{\def\PY@tc##1{\textcolor[rgb]{0.73,0.73,0.73}{##1}}}
\expandafter\def\csname PY@tok@kt\endcsname{\def\PY@tc##1{\textcolor[rgb]{0.69,0.00,0.25}{##1}}}
\expandafter\def\csname PY@tok@ow\endcsname{\let\PY@bf=\textbf\def\PY@tc##1{\textcolor[rgb]{0.67,0.13,1.00}{##1}}}
\expandafter\def\csname PY@tok@sb\endcsname{\def\PY@tc##1{\textcolor[rgb]{0.73,0.13,0.13}{##1}}}
\expandafter\def\csname PY@tok@k\endcsname{\let\PY@bf=\textbf\def\PY@tc##1{\textcolor[rgb]{0.00,0.50,0.00}{##1}}}
\expandafter\def\csname PY@tok@se\endcsname{\let\PY@bf=\textbf\def\PY@tc##1{\textcolor[rgb]{0.73,0.40,0.13}{##1}}}
\expandafter\def\csname PY@tok@sd\endcsname{\let\PY@it=\textit\def\PY@tc##1{\textcolor[rgb]{0.73,0.13,0.13}{##1}}}

\def\PYZbs{\char`\\}
\def\PYZus{\char`\_}
\def\PYZob{\char`\{}
\def\PYZcb{\char`\}}
\def\PYZca{\char`\^}
\def\PYZam{\char`\&}
\def\PYZlt{\char`\<}
\def\PYZgt{\char`\>}
\def\PYZsh{\char`\#}
\def\PYZpc{\char`\%}
\def\PYZdl{\char`\$}
\def\PYZhy{\char`\-}
\def\PYZsq{\char`\'}
\def\PYZdq{\char`\"}
\def\PYZti{\char`\~}
% for compatibility with earlier versions
\def\PYZat{@}
\def\PYZlb{[}
\def\PYZrb{]}
\makeatother


    % Exact colors from NB
    \definecolor{incolor}{rgb}{0.0, 0.0, 0.5}
    \definecolor{outcolor}{rgb}{0.545, 0.0, 0.0}



    
    % Prevent overflowing lines due to hard-to-break entities
    \sloppy 
    % Setup hyperref package
    \hypersetup{
      breaklinks=true,  % so long urls are correctly broken across lines
      colorlinks=true,
      urlcolor=blue,
      linkcolor=darkorange,
      citecolor=darkgreen,
      }
    % Slightly bigger margins than the latex defaults
    
     

    \begin{document}
    
    
    \maketitle
    
    

    
    \section{Classes}\label{classes}

    Variables, Lists, Dictionaries etc in python is a object. Without
getting into the theory part of Object Oriented Programming, explanation
of the concepts will be done along this tutorial.

    A class is declared as follows

    class class\_name:

\begin{verbatim}
Functions
\end{verbatim}

    \begin{Verbatim}[commandchars=\\\{\}]
{\color{incolor}In [{\color{incolor}1}]:} \PY{k}{class} \PY{n+nc}{FirstClass}\PY{p}{:}
            \PY{k}{pass}
\end{Verbatim}

    \textbf{pass} in python means do nothing.

    Above, a class object named ``FirstClass'' is declared now consider a
``egclass'' which has all the characteristics of ``FirstClass''. So all
you have to do is, equate the ``egclass'' to ``FirstClass''. In python
jargon this is called as creating an instance. ``egclass'' is the
instance of ``FirstClass''

    \begin{Verbatim}[commandchars=\\\{\}]
{\color{incolor}In [{\color{incolor}2}]:} \PY{n}{egclass} \PY{o}{=} \PY{n}{FirstClass}\PY{p}{(}\PY{p}{)}
\end{Verbatim}

    \begin{Verbatim}[commandchars=\\\{\}]
{\color{incolor}In [{\color{incolor}3}]:} \PY{n+nb}{type}\PY{p}{(}\PY{n}{egclass}\PY{p}{)}
\end{Verbatim}

            \begin{Verbatim}[commandchars=\\\{\}]
{\color{outcolor}Out[{\color{outcolor}3}]:} instance
\end{Verbatim}
        
    \begin{Verbatim}[commandchars=\\\{\}]
{\color{incolor}In [{\color{incolor}4}]:} \PY{n+nb}{type}\PY{p}{(}\PY{n}{FirstClass}\PY{p}{)}
\end{Verbatim}

            \begin{Verbatim}[commandchars=\\\{\}]
{\color{outcolor}Out[{\color{outcolor}4}]:} classobj
\end{Verbatim}
        
    Now let us add some ``functionality'' to the class. So that our
``FirstClass'' is defined in a better way. A function inside a class is
called as a ``Method'' of that class

    Most of the classes will have a function named ``\_\_init\_\_''. These
are called as magic methods. In this method you basically initialize the
variables of that class or any other initial algorithms which is
applicable to all methods is specified in this method. A variable inside
a class is called an attribute.

    These helps simplify the process of initializing a instance. For
example,

Without the use of magic method or \_\_init\_\_ which is otherwise
called as constructors. One had to define a \textbf{init( )} method and
call the \textbf{init( )} function.

    \begin{Verbatim}[commandchars=\\\{\}]
{\color{incolor}In [{\color{incolor} }]:} \PY{n}{eg0} \PY{o}{=} \PY{n}{FirstClass}\PY{p}{(}\PY{p}{)}
        \PY{n}{eg0}\PY{o}{.}\PY{n}{init}\PY{p}{(}\PY{p}{)}
\end{Verbatim}

    But when the constructor is defined the \_\_init\_\_ is called thus
intializing the instance created.

    We will make our ``FirstClass'' to accept two variables name and symbol.

I will be explaining about the ``self'' in a while.

    \begin{Verbatim}[commandchars=\\\{\}]
{\color{incolor}In [{\color{incolor}6}]:} \PY{k}{class} \PY{n+nc}{FirstClass}\PY{p}{:}
            \PY{k}{def} \PY{n+nf}{\PYZus{}\PYZus{}init\PYZus{}\PYZus{}}\PY{p}{(}\PY{n+nb+bp}{self}\PY{p}{,}\PY{n}{name}\PY{p}{,}\PY{n}{symbol}\PY{p}{)}\PY{p}{:}
                \PY{n+nb+bp}{self}\PY{o}{.}\PY{n}{name} \PY{o}{=} \PY{n}{name}
                \PY{n+nb+bp}{self}\PY{o}{.}\PY{n}{symbol} \PY{o}{=} \PY{n}{symbol}
\end{Verbatim}

    Now that we have defined a function and added the \_\_init\_\_ method.
We can create a instance of FirstClass which now accepts two arguments.

    \begin{Verbatim}[commandchars=\\\{\}]
{\color{incolor}In [{\color{incolor}7}]:} \PY{n}{eg1} \PY{o}{=} \PY{n}{FirstClass}\PY{p}{(}\PY{l+s}{\PYZsq{}}\PY{l+s}{one}\PY{l+s}{\PYZsq{}}\PY{p}{,}\PY{l+m+mi}{1}\PY{p}{)}
        \PY{n}{eg2} \PY{o}{=} \PY{n}{FirstClass}\PY{p}{(}\PY{l+s}{\PYZsq{}}\PY{l+s}{two}\PY{l+s}{\PYZsq{}}\PY{p}{,}\PY{l+m+mi}{2}\PY{p}{)}
\end{Verbatim}

    \begin{Verbatim}[commandchars=\\\{\}]
{\color{incolor}In [{\color{incolor}8}]:} \PY{k}{print} \PY{n}{eg1}\PY{o}{.}\PY{n}{name}\PY{p}{,} \PY{n}{eg1}\PY{o}{.}\PY{n}{symbol}
        \PY{k}{print} \PY{n}{eg2}\PY{o}{.}\PY{n}{name}\PY{p}{,} \PY{n}{eg2}\PY{o}{.}\PY{n}{symbol}
\end{Verbatim}

    \begin{Verbatim}[commandchars=\\\{\}]
one 1
two 2
    \end{Verbatim}

    \textbf{dir( )} function comes very handy in looking into what the class
contains and what all method it offers

    \begin{Verbatim}[commandchars=\\\{\}]
{\color{incolor}In [{\color{incolor}9}]:} \PY{n+nb}{dir}\PY{p}{(}\PY{n}{FirstClass}\PY{p}{)}
\end{Verbatim}

            \begin{Verbatim}[commandchars=\\\{\}]
{\color{outcolor}Out[{\color{outcolor}9}]:} ['\_\_doc\_\_', '\_\_init\_\_', '\_\_module\_\_']
\end{Verbatim}
        
    \textbf{dir( )} of an instance also shows it's defined attributes.

    \begin{Verbatim}[commandchars=\\\{\}]
{\color{incolor}In [{\color{incolor}10}]:} \PY{n+nb}{dir}\PY{p}{(}\PY{n}{eg1}\PY{p}{)}
\end{Verbatim}

            \begin{Verbatim}[commandchars=\\\{\}]
{\color{outcolor}Out[{\color{outcolor}10}]:} ['\_\_doc\_\_', '\_\_init\_\_', '\_\_module\_\_', 'name', 'symbol']
\end{Verbatim}
        
    Changing the FirstClass function a bit,

    \begin{Verbatim}[commandchars=\\\{\}]
{\color{incolor}In [{\color{incolor}11}]:} \PY{k}{class} \PY{n+nc}{FirstClass}\PY{p}{:}
             \PY{k}{def} \PY{n+nf}{\PYZus{}\PYZus{}init\PYZus{}\PYZus{}}\PY{p}{(}\PY{n+nb+bp}{self}\PY{p}{,}\PY{n}{name}\PY{p}{,}\PY{n}{symbol}\PY{p}{)}\PY{p}{:}
                 \PY{n+nb+bp}{self}\PY{o}{.}\PY{n}{n} \PY{o}{=} \PY{n}{name}
                 \PY{n+nb+bp}{self}\PY{o}{.}\PY{n}{s} \PY{o}{=} \PY{n}{symbol}
\end{Verbatim}

    Changing self.name and self.symbol to self.n and self.s respectively
will yield,

    \begin{Verbatim}[commandchars=\\\{\}]
{\color{incolor}In [{\color{incolor}12}]:} \PY{n}{eg1} \PY{o}{=} \PY{n}{FirstClass}\PY{p}{(}\PY{l+s}{\PYZsq{}}\PY{l+s}{one}\PY{l+s}{\PYZsq{}}\PY{p}{,}\PY{l+m+mi}{1}\PY{p}{)}
         \PY{n}{eg2} \PY{o}{=} \PY{n}{FirstClass}\PY{p}{(}\PY{l+s}{\PYZsq{}}\PY{l+s}{two}\PY{l+s}{\PYZsq{}}\PY{p}{,}\PY{l+m+mi}{2}\PY{p}{)}
\end{Verbatim}

    \begin{Verbatim}[commandchars=\\\{\}]
{\color{incolor}In [{\color{incolor}13}]:} \PY{k}{print} \PY{n}{eg1}\PY{o}{.}\PY{n}{name}\PY{p}{,} \PY{n}{eg1}\PY{o}{.}\PY{n}{symbol}
         \PY{k}{print} \PY{n}{eg2}\PY{o}{.}\PY{n}{name}\PY{p}{,} \PY{n}{eg2}\PY{o}{.}\PY{n}{symbol}
\end{Verbatim}

    \begin{Verbatim}[commandchars=\\\{\}]

        ---------------------------------------------------------------------------

        AttributeError                            Traceback (most recent call last)

        <ipython-input-13-3717d682d1cf> in <module>()
    ----> 1 print eg1.name, eg1.symbol
          2 print eg2.name, eg2.symbol


        AttributeError: FirstClass instance has no attribute 'name'

    \end{Verbatim}

    AttributeError, Remember variables are nothing but attributes inside a
class? So this means we have not given the correct attribute for the
instance.

    \begin{Verbatim}[commandchars=\\\{\}]
{\color{incolor}In [{\color{incolor}14}]:} \PY{n+nb}{dir}\PY{p}{(}\PY{n}{eg1}\PY{p}{)}
\end{Verbatim}

            \begin{Verbatim}[commandchars=\\\{\}]
{\color{outcolor}Out[{\color{outcolor}14}]:} ['\_\_doc\_\_', '\_\_init\_\_', '\_\_module\_\_', 'n', 's']
\end{Verbatim}
        
    \begin{Verbatim}[commandchars=\\\{\}]
{\color{incolor}In [{\color{incolor}15}]:} \PY{k}{print} \PY{n}{eg1}\PY{o}{.}\PY{n}{n}\PY{p}{,} \PY{n}{eg1}\PY{o}{.}\PY{n}{s}
         \PY{k}{print} \PY{n}{eg2}\PY{o}{.}\PY{n}{n}\PY{p}{,} \PY{n}{eg2}\PY{o}{.}\PY{n}{s}
\end{Verbatim}

    \begin{Verbatim}[commandchars=\\\{\}]
one 1
two 2
    \end{Verbatim}

    So now we have solved the error. Now let us compare the two examples
that we saw.

When I declared self.name and self.symbol, there was no attribute error
for eg1.name and eg1.symbol and when I declared self.n and self.s, there
was no attribute error for eg1.n and eg1.s

From the above we can conclude that self is nothing but the instance
itself.

Remember, self is not predefined it is userdefined. You can make use of
anything you are comfortable with. But it has become a common practice
to use self.

    \begin{Verbatim}[commandchars=\\\{\}]
{\color{incolor}In [{\color{incolor}16}]:} \PY{k}{class} \PY{n+nc}{FirstClass}\PY{p}{:}
             \PY{k}{def} \PY{n+nf}{\PYZus{}\PYZus{}init\PYZus{}\PYZus{}}\PY{p}{(}\PY{n}{asdf1234}\PY{p}{,}\PY{n}{name}\PY{p}{,}\PY{n}{symbol}\PY{p}{)}\PY{p}{:}
                 \PY{n}{asdf1234}\PY{o}{.}\PY{n}{n} \PY{o}{=} \PY{n}{name}
                 \PY{n}{asdf1234}\PY{o}{.}\PY{n}{s} \PY{o}{=} \PY{n}{symbol}
\end{Verbatim}

    \begin{Verbatim}[commandchars=\\\{\}]
{\color{incolor}In [{\color{incolor}17}]:} \PY{n}{eg1} \PY{o}{=} \PY{n}{FirstClass}\PY{p}{(}\PY{l+s}{\PYZsq{}}\PY{l+s}{one}\PY{l+s}{\PYZsq{}}\PY{p}{,}\PY{l+m+mi}{1}\PY{p}{)}
         \PY{n}{eg2} \PY{o}{=} \PY{n}{FirstClass}\PY{p}{(}\PY{l+s}{\PYZsq{}}\PY{l+s}{two}\PY{l+s}{\PYZsq{}}\PY{p}{,}\PY{l+m+mi}{2}\PY{p}{)}
\end{Verbatim}

    \begin{Verbatim}[commandchars=\\\{\}]
{\color{incolor}In [{\color{incolor}18}]:} \PY{k}{print} \PY{n}{eg1}\PY{o}{.}\PY{n}{n}\PY{p}{,} \PY{n}{eg1}\PY{o}{.}\PY{n}{s}
         \PY{k}{print} \PY{n}{eg2}\PY{o}{.}\PY{n}{n}\PY{p}{,} \PY{n}{eg2}\PY{o}{.}\PY{n}{s}
\end{Verbatim}

    \begin{Verbatim}[commandchars=\\\{\}]
one 1
two 2
    \end{Verbatim}

    Since eg1 and eg2 are instances of FirstClass it need not necessarily be
limited to FirstClass itself. It might extend itself by declaring other
attributes without having the attribute to be declared inside the
FirstClass.

    \begin{Verbatim}[commandchars=\\\{\}]
{\color{incolor}In [{\color{incolor}19}]:} \PY{n}{eg1}\PY{o}{.}\PY{n}{cube} \PY{o}{=} \PY{l+m+mi}{1}
         \PY{n}{eg2}\PY{o}{.}\PY{n}{cube} \PY{o}{=} \PY{l+m+mi}{8}
\end{Verbatim}

    \begin{Verbatim}[commandchars=\\\{\}]
{\color{incolor}In [{\color{incolor}20}]:} \PY{n+nb}{dir}\PY{p}{(}\PY{n}{eg1}\PY{p}{)}
\end{Verbatim}

            \begin{Verbatim}[commandchars=\\\{\}]
{\color{outcolor}Out[{\color{outcolor}20}]:} ['\_\_doc\_\_', '\_\_init\_\_', '\_\_module\_\_', 'cube', 'n', 's']
\end{Verbatim}
        
    Just like global and local variables as we saw earlier, even classes
have it's own types of variables.

Class Attribute : attributes defined outside the method and is
applicable to all the instances.

Instance Attribute : attributes defined inside a method and is
applicable to only that method and is unique to each instance.

    \begin{Verbatim}[commandchars=\\\{\}]
{\color{incolor}In [{\color{incolor}21}]:} \PY{k}{class} \PY{n+nc}{FirstClass}\PY{p}{:}
             \PY{n}{test} \PY{o}{=} \PY{l+s}{\PYZsq{}}\PY{l+s}{test}\PY{l+s}{\PYZsq{}}
             \PY{k}{def} \PY{n+nf}{\PYZus{}\PYZus{}init\PYZus{}\PYZus{}}\PY{p}{(}\PY{n+nb+bp}{self}\PY{p}{,}\PY{n}{name}\PY{p}{,}\PY{n}{symbol}\PY{p}{)}\PY{p}{:}
                 \PY{n+nb+bp}{self}\PY{o}{.}\PY{n}{name} \PY{o}{=} \PY{n}{name}
                 \PY{n+nb+bp}{self}\PY{o}{.}\PY{n}{symbol} \PY{o}{=} \PY{n}{symbol}
\end{Verbatim}

    Here test is a class attribute and name is a instance attribute.

    \begin{Verbatim}[commandchars=\\\{\}]
{\color{incolor}In [{\color{incolor}22}]:} \PY{n}{eg3} \PY{o}{=} \PY{n}{FirstClass}\PY{p}{(}\PY{l+s}{\PYZsq{}}\PY{l+s}{Three}\PY{l+s}{\PYZsq{}}\PY{p}{,}\PY{l+m+mi}{3}\PY{p}{)}
\end{Verbatim}

    \begin{Verbatim}[commandchars=\\\{\}]
{\color{incolor}In [{\color{incolor}23}]:} \PY{k}{print} \PY{n}{eg3}\PY{o}{.}\PY{n}{test}\PY{p}{,} \PY{n}{eg3}\PY{o}{.}\PY{n}{name}
\end{Verbatim}

    \begin{Verbatim}[commandchars=\\\{\}]
test Three
    \end{Verbatim}

    Let us add some more methods to FirstClass.

    \begin{Verbatim}[commandchars=\\\{\}]
{\color{incolor}In [{\color{incolor}24}]:} \PY{k}{class} \PY{n+nc}{FirstClass}\PY{p}{:}
             \PY{k}{def} \PY{n+nf}{\PYZus{}\PYZus{}init\PYZus{}\PYZus{}}\PY{p}{(}\PY{n+nb+bp}{self}\PY{p}{,}\PY{n}{name}\PY{p}{,}\PY{n}{symbol}\PY{p}{)}\PY{p}{:}
                 \PY{n+nb+bp}{self}\PY{o}{.}\PY{n}{name} \PY{o}{=} \PY{n}{name}
                 \PY{n+nb+bp}{self}\PY{o}{.}\PY{n}{symbol} \PY{o}{=} \PY{n}{symbol}
             \PY{k}{def} \PY{n+nf}{square}\PY{p}{(}\PY{n+nb+bp}{self}\PY{p}{)}\PY{p}{:}
                 \PY{k}{return} \PY{n+nb+bp}{self}\PY{o}{.}\PY{n}{symbol} \PY{o}{*} \PY{n+nb+bp}{self}\PY{o}{.}\PY{n}{symbol}
             \PY{k}{def} \PY{n+nf}{cube}\PY{p}{(}\PY{n+nb+bp}{self}\PY{p}{)}\PY{p}{:}
                 \PY{k}{return} \PY{n+nb+bp}{self}\PY{o}{.}\PY{n}{symbol} \PY{o}{*} \PY{n+nb+bp}{self}\PY{o}{.}\PY{n}{symbol} \PY{o}{*} \PY{n+nb+bp}{self}\PY{o}{.}\PY{n}{symbol}
             \PY{k}{def} \PY{n+nf}{multiply}\PY{p}{(}\PY{n+nb+bp}{self}\PY{p}{,} \PY{n}{x}\PY{p}{)}\PY{p}{:}
                 \PY{k}{return} \PY{n+nb+bp}{self}\PY{o}{.}\PY{n}{symbol} \PY{o}{*} \PY{n}{x}
\end{Verbatim}

    \begin{Verbatim}[commandchars=\\\{\}]
{\color{incolor}In [{\color{incolor}25}]:} \PY{n}{eg4} \PY{o}{=} \PY{n}{FirstClass}\PY{p}{(}\PY{l+s}{\PYZsq{}}\PY{l+s}{Five}\PY{l+s}{\PYZsq{}}\PY{p}{,}\PY{l+m+mi}{5}\PY{p}{)}
\end{Verbatim}

    \begin{Verbatim}[commandchars=\\\{\}]
{\color{incolor}In [{\color{incolor}26}]:} \PY{k}{print} \PY{n}{eg4}\PY{o}{.}\PY{n}{square}\PY{p}{(}\PY{p}{)}
         \PY{k}{print} \PY{n}{eg4}\PY{o}{.}\PY{n}{cube}\PY{p}{(}\PY{p}{)}
\end{Verbatim}

    \begin{Verbatim}[commandchars=\\\{\}]
25
125
    \end{Verbatim}

    \begin{Verbatim}[commandchars=\\\{\}]
{\color{incolor}In [{\color{incolor}27}]:} \PY{n}{eg4}\PY{o}{.}\PY{n}{multiply}\PY{p}{(}\PY{l+m+mi}{2}\PY{p}{)}
\end{Verbatim}

            \begin{Verbatim}[commandchars=\\\{\}]
{\color{outcolor}Out[{\color{outcolor}27}]:} 10
\end{Verbatim}
        
    The above can also be written as,

    \begin{Verbatim}[commandchars=\\\{\}]
{\color{incolor}In [{\color{incolor}28}]:} \PY{n}{FirstClass}\PY{o}{.}\PY{n}{multiply}\PY{p}{(}\PY{n}{eg4}\PY{p}{,}\PY{l+m+mi}{2}\PY{p}{)}
\end{Verbatim}

            \begin{Verbatim}[commandchars=\\\{\}]
{\color{outcolor}Out[{\color{outcolor}28}]:} 10
\end{Verbatim}
        
    \subsection{Inheritance}\label{inheritance}

    There might be cases where a new class would have all the previous
characteristics of an already defined class. So the new class can
``inherit'' the previous class and add it's own methods to it. This is
called as inheritance.

    Consider class SoftwareEngineer which has a method salary.

    \begin{Verbatim}[commandchars=\\\{\}]
{\color{incolor}In [{\color{incolor}29}]:} \PY{k}{class} \PY{n+nc}{SoftwareEngineer}\PY{p}{:}
             \PY{k}{def} \PY{n+nf}{\PYZus{}\PYZus{}init\PYZus{}\PYZus{}}\PY{p}{(}\PY{n+nb+bp}{self}\PY{p}{,}\PY{n}{name}\PY{p}{,}\PY{n}{age}\PY{p}{)}\PY{p}{:}
                 \PY{n+nb+bp}{self}\PY{o}{.}\PY{n}{name} \PY{o}{=} \PY{n}{name}
                 \PY{n+nb+bp}{self}\PY{o}{.}\PY{n}{age} \PY{o}{=} \PY{n}{age}
             \PY{k}{def} \PY{n+nf}{salary}\PY{p}{(}\PY{n+nb+bp}{self}\PY{p}{,} \PY{n}{value}\PY{p}{)}\PY{p}{:}
                 \PY{n+nb+bp}{self}\PY{o}{.}\PY{n}{money} \PY{o}{=} \PY{n}{value}
                 \PY{k}{print} \PY{n+nb+bp}{self}\PY{o}{.}\PY{n}{name}\PY{p}{,}\PY{l+s}{\PYZdq{}}\PY{l+s}{earns}\PY{l+s}{\PYZdq{}}\PY{p}{,}\PY{n+nb+bp}{self}\PY{o}{.}\PY{n}{money}
\end{Verbatim}

    \begin{Verbatim}[commandchars=\\\{\}]
{\color{incolor}In [{\color{incolor}30}]:} \PY{n}{a} \PY{o}{=} \PY{n}{SoftwareEngineer}\PY{p}{(}\PY{l+s}{\PYZsq{}}\PY{l+s}{Kartik}\PY{l+s}{\PYZsq{}}\PY{p}{,}\PY{l+m+mi}{26}\PY{p}{)}
\end{Verbatim}

    \begin{Verbatim}[commandchars=\\\{\}]
{\color{incolor}In [{\color{incolor}31}]:} \PY{n}{a}\PY{o}{.}\PY{n}{salary}\PY{p}{(}\PY{l+m+mi}{40000}\PY{p}{)}
\end{Verbatim}

    \begin{Verbatim}[commandchars=\\\{\}]
Kartik earns 40000
    \end{Verbatim}

    \begin{Verbatim}[commandchars=\\\{\}]
{\color{incolor}In [{\color{incolor}32}]:} \PY{n+nb}{dir}\PY{p}{(}\PY{n}{SoftwareEngineer}\PY{p}{)}
\end{Verbatim}

            \begin{Verbatim}[commandchars=\\\{\}]
{\color{outcolor}Out[{\color{outcolor}32}]:} ['\_\_doc\_\_', '\_\_init\_\_', '\_\_module\_\_', 'salary']
\end{Verbatim}
        
    Now consider another class Artist which tells us about the amount of
money an artist earns and his artform.

    \begin{Verbatim}[commandchars=\\\{\}]
{\color{incolor}In [{\color{incolor}33}]:} \PY{k}{class} \PY{n+nc}{Artist}\PY{p}{:}
             \PY{k}{def} \PY{n+nf}{\PYZus{}\PYZus{}init\PYZus{}\PYZus{}}\PY{p}{(}\PY{n+nb+bp}{self}\PY{p}{,}\PY{n}{name}\PY{p}{,}\PY{n}{age}\PY{p}{)}\PY{p}{:}
                 \PY{n+nb+bp}{self}\PY{o}{.}\PY{n}{name} \PY{o}{=} \PY{n}{name}
                 \PY{n+nb+bp}{self}\PY{o}{.}\PY{n}{age} \PY{o}{=} \PY{n}{age}
             \PY{k}{def} \PY{n+nf}{money}\PY{p}{(}\PY{n+nb+bp}{self}\PY{p}{,}\PY{n}{value}\PY{p}{)}\PY{p}{:}
                 \PY{n+nb+bp}{self}\PY{o}{.}\PY{n}{money} \PY{o}{=} \PY{n}{value}
                 \PY{k}{print} \PY{n+nb+bp}{self}\PY{o}{.}\PY{n}{name}\PY{p}{,}\PY{l+s}{\PYZdq{}}\PY{l+s}{earns}\PY{l+s}{\PYZdq{}}\PY{p}{,}\PY{n+nb+bp}{self}\PY{o}{.}\PY{n}{money}
             \PY{k}{def} \PY{n+nf}{artform}\PY{p}{(}\PY{n+nb+bp}{self}\PY{p}{,} \PY{n}{job}\PY{p}{)}\PY{p}{:}
                 \PY{n+nb+bp}{self}\PY{o}{.}\PY{n}{job} \PY{o}{=} \PY{n}{job}
                 \PY{k}{print} \PY{n+nb+bp}{self}\PY{o}{.}\PY{n}{name}\PY{p}{,}\PY{l+s}{\PYZdq{}}\PY{l+s}{is a}\PY{l+s}{\PYZdq{}}\PY{p}{,} \PY{n+nb+bp}{self}\PY{o}{.}\PY{n}{job}
\end{Verbatim}

    \begin{Verbatim}[commandchars=\\\{\}]
{\color{incolor}In [{\color{incolor}34}]:} \PY{n}{b} \PY{o}{=} \PY{n}{Artist}\PY{p}{(}\PY{l+s}{\PYZsq{}}\PY{l+s}{Nitin}\PY{l+s}{\PYZsq{}}\PY{p}{,}\PY{l+m+mi}{20}\PY{p}{)}
\end{Verbatim}

    \begin{Verbatim}[commandchars=\\\{\}]
{\color{incolor}In [{\color{incolor}35}]:} \PY{n}{b}\PY{o}{.}\PY{n}{money}\PY{p}{(}\PY{l+m+mi}{50000}\PY{p}{)}
         \PY{n}{b}\PY{o}{.}\PY{n}{artform}\PY{p}{(}\PY{l+s}{\PYZsq{}}\PY{l+s}{Musician}\PY{l+s}{\PYZsq{}}\PY{p}{)}
\end{Verbatim}

    \begin{Verbatim}[commandchars=\\\{\}]
Nitin earns 50000
Nitin is a Musician
    \end{Verbatim}

    \begin{Verbatim}[commandchars=\\\{\}]
{\color{incolor}In [{\color{incolor}36}]:} \PY{n+nb}{dir}\PY{p}{(}\PY{n}{Artist}\PY{p}{)}
\end{Verbatim}

            \begin{Verbatim}[commandchars=\\\{\}]
{\color{outcolor}Out[{\color{outcolor}36}]:} ['\_\_doc\_\_', '\_\_init\_\_', '\_\_module\_\_', 'artform', 'money']
\end{Verbatim}
        
    money method and salary method are the same. So we can generalize the
method to salary and inherit the SoftwareEngineer class to Artist class.
Now the artist class becomes,

    \begin{Verbatim}[commandchars=\\\{\}]
{\color{incolor}In [{\color{incolor}37}]:} \PY{k}{class} \PY{n+nc}{Artist}\PY{p}{(}\PY{n}{SoftwareEngineer}\PY{p}{)}\PY{p}{:}
             \PY{k}{def} \PY{n+nf}{artform}\PY{p}{(}\PY{n+nb+bp}{self}\PY{p}{,} \PY{n}{job}\PY{p}{)}\PY{p}{:}
                 \PY{n+nb+bp}{self}\PY{o}{.}\PY{n}{job} \PY{o}{=} \PY{n}{job}
                 \PY{k}{print} \PY{n+nb+bp}{self}\PY{o}{.}\PY{n}{name}\PY{p}{,}\PY{l+s}{\PYZdq{}}\PY{l+s}{is a}\PY{l+s}{\PYZdq{}}\PY{p}{,} \PY{n+nb+bp}{self}\PY{o}{.}\PY{n}{job}
\end{Verbatim}

    \begin{Verbatim}[commandchars=\\\{\}]
{\color{incolor}In [{\color{incolor}38}]:} \PY{n}{c} \PY{o}{=} \PY{n}{Artist}\PY{p}{(}\PY{l+s}{\PYZsq{}}\PY{l+s}{Nishanth}\PY{l+s}{\PYZsq{}}\PY{p}{,}\PY{l+m+mi}{21}\PY{p}{)}
\end{Verbatim}

    \begin{Verbatim}[commandchars=\\\{\}]
{\color{incolor}In [{\color{incolor}39}]:} \PY{n+nb}{dir}\PY{p}{(}\PY{n}{Artist}\PY{p}{)}
\end{Verbatim}

            \begin{Verbatim}[commandchars=\\\{\}]
{\color{outcolor}Out[{\color{outcolor}39}]:} ['\_\_doc\_\_', '\_\_init\_\_', '\_\_module\_\_', 'artform', 'salary']
\end{Verbatim}
        
    \begin{Verbatim}[commandchars=\\\{\}]
{\color{incolor}In [{\color{incolor}40}]:} \PY{n}{c}\PY{o}{.}\PY{n}{salary}\PY{p}{(}\PY{l+m+mi}{60000}\PY{p}{)}
         \PY{n}{c}\PY{o}{.}\PY{n}{artform}\PY{p}{(}\PY{l+s}{\PYZsq{}}\PY{l+s}{Dancer}\PY{l+s}{\PYZsq{}}\PY{p}{)}
\end{Verbatim}

    \begin{Verbatim}[commandchars=\\\{\}]
Nishanth earns 60000
Nishanth is a Dancer
    \end{Verbatim}

    Suppose say while inheriting a particular method is not suitable for the
new class. One can override this method by defining again that method
with the same name inside the new class.

    \begin{Verbatim}[commandchars=\\\{\}]
{\color{incolor}In [{\color{incolor}41}]:} \PY{k}{class} \PY{n+nc}{Artist}\PY{p}{(}\PY{n}{SoftwareEngineer}\PY{p}{)}\PY{p}{:}
             \PY{k}{def} \PY{n+nf}{artform}\PY{p}{(}\PY{n+nb+bp}{self}\PY{p}{,} \PY{n}{job}\PY{p}{)}\PY{p}{:}
                 \PY{n+nb+bp}{self}\PY{o}{.}\PY{n}{job} \PY{o}{=} \PY{n}{job}
                 \PY{k}{print} \PY{n+nb+bp}{self}\PY{o}{.}\PY{n}{name}\PY{p}{,}\PY{l+s}{\PYZdq{}}\PY{l+s}{is a}\PY{l+s}{\PYZdq{}}\PY{p}{,} \PY{n+nb+bp}{self}\PY{o}{.}\PY{n}{job}
             \PY{k}{def} \PY{n+nf}{salary}\PY{p}{(}\PY{n+nb+bp}{self}\PY{p}{,} \PY{n}{value}\PY{p}{)}\PY{p}{:}
                 \PY{n+nb+bp}{self}\PY{o}{.}\PY{n}{money} \PY{o}{=} \PY{n}{value}
                 \PY{k}{print} \PY{n+nb+bp}{self}\PY{o}{.}\PY{n}{name}\PY{p}{,}\PY{l+s}{\PYZdq{}}\PY{l+s}{earns}\PY{l+s}{\PYZdq{}}\PY{p}{,}\PY{n+nb+bp}{self}\PY{o}{.}\PY{n}{money}
                 \PY{k}{print} \PY{l+s}{\PYZdq{}}\PY{l+s}{I am overriding the SoftwareEngineer class}\PY{l+s}{\PYZsq{}}\PY{l+s}{s salary method}\PY{l+s}{\PYZdq{}}
\end{Verbatim}

    \begin{Verbatim}[commandchars=\\\{\}]
{\color{incolor}In [{\color{incolor}42}]:} \PY{n}{c} \PY{o}{=} \PY{n}{Artist}\PY{p}{(}\PY{l+s}{\PYZsq{}}\PY{l+s}{Nishanth}\PY{l+s}{\PYZsq{}}\PY{p}{,}\PY{l+m+mi}{21}\PY{p}{)}
\end{Verbatim}

    \begin{Verbatim}[commandchars=\\\{\}]
{\color{incolor}In [{\color{incolor}43}]:} \PY{n}{c}\PY{o}{.}\PY{n}{salary}\PY{p}{(}\PY{l+m+mi}{60000}\PY{p}{)}
         \PY{n}{c}\PY{o}{.}\PY{n}{artform}\PY{p}{(}\PY{l+s}{\PYZsq{}}\PY{l+s}{Dancer}\PY{l+s}{\PYZsq{}}\PY{p}{)}
\end{Verbatim}

    \begin{Verbatim}[commandchars=\\\{\}]
Nishanth earns 60000
I am overriding the SoftwareEngineer class's salary method
Nishanth is a Dancer
    \end{Verbatim}

    If not sure how many times methods will be called it will become
difficult to declare so many variables to carry each result hence it is
better to declare a list and append the result.

    \begin{Verbatim}[commandchars=\\\{\}]
{\color{incolor}In [{\color{incolor}44}]:} \PY{k}{class} \PY{n+nc}{emptylist}\PY{p}{:}
             \PY{k}{def} \PY{n+nf}{\PYZus{}\PYZus{}init\PYZus{}\PYZus{}}\PY{p}{(}\PY{n+nb+bp}{self}\PY{p}{)}\PY{p}{:}
                 \PY{n+nb+bp}{self}\PY{o}{.}\PY{n}{data} \PY{o}{=} \PY{p}{[}\PY{p}{]}
             \PY{k}{def} \PY{n+nf}{one}\PY{p}{(}\PY{n+nb+bp}{self}\PY{p}{,}\PY{n}{x}\PY{p}{)}\PY{p}{:}
                 \PY{n+nb+bp}{self}\PY{o}{.}\PY{n}{data}\PY{o}{.}\PY{n}{append}\PY{p}{(}\PY{n}{x}\PY{p}{)}
             \PY{k}{def} \PY{n+nf}{two}\PY{p}{(}\PY{n+nb+bp}{self}\PY{p}{,} \PY{n}{x} \PY{p}{)}\PY{p}{:}
                 \PY{n+nb+bp}{self}\PY{o}{.}\PY{n}{data}\PY{o}{.}\PY{n}{append}\PY{p}{(}\PY{n}{x}\PY{o}{*}\PY{o}{*}\PY{l+m+mi}{2}\PY{p}{)}
             \PY{k}{def} \PY{n+nf}{three}\PY{p}{(}\PY{n+nb+bp}{self}\PY{p}{,} \PY{n}{x}\PY{p}{)}\PY{p}{:}
                 \PY{n+nb+bp}{self}\PY{o}{.}\PY{n}{data}\PY{o}{.}\PY{n}{append}\PY{p}{(}\PY{n}{x}\PY{o}{*}\PY{o}{*}\PY{l+m+mi}{3}\PY{p}{)}
\end{Verbatim}

    \begin{Verbatim}[commandchars=\\\{\}]
{\color{incolor}In [{\color{incolor}45}]:} \PY{n}{xc} \PY{o}{=} \PY{n}{emptylist}\PY{p}{(}\PY{p}{)}
\end{Verbatim}

    \begin{Verbatim}[commandchars=\\\{\}]
{\color{incolor}In [{\color{incolor}46}]:} \PY{n}{xc}\PY{o}{.}\PY{n}{one}\PY{p}{(}\PY{l+m+mi}{1}\PY{p}{)}
         \PY{k}{print} \PY{n}{xc}\PY{o}{.}\PY{n}{data}
\end{Verbatim}

    \begin{Verbatim}[commandchars=\\\{\}]
[1]
    \end{Verbatim}

    Since xc.data is a list direct list operations can also be performed.

    \begin{Verbatim}[commandchars=\\\{\}]
{\color{incolor}In [{\color{incolor}47}]:} \PY{n}{xc}\PY{o}{.}\PY{n}{data}\PY{o}{.}\PY{n}{append}\PY{p}{(}\PY{l+m+mi}{8}\PY{p}{)}
         \PY{k}{print} \PY{n}{xc}\PY{o}{.}\PY{n}{data}
\end{Verbatim}

    \begin{Verbatim}[commandchars=\\\{\}]
[1, 8]
    \end{Verbatim}

    \begin{Verbatim}[commandchars=\\\{\}]
{\color{incolor}In [{\color{incolor}48}]:} \PY{n}{xc}\PY{o}{.}\PY{n}{two}\PY{p}{(}\PY{l+m+mi}{3}\PY{p}{)}
         \PY{k}{print} \PY{n}{xc}\PY{o}{.}\PY{n}{data}
\end{Verbatim}

    \begin{Verbatim}[commandchars=\\\{\}]
[1, 8, 9]
    \end{Verbatim}

    If the number of input arguments varies from instance to instance
asterisk can be used as shown.

    \begin{Verbatim}[commandchars=\\\{\}]
{\color{incolor}In [{\color{incolor}49}]:} \PY{k}{class} \PY{n+nc}{NotSure}\PY{p}{:}
             \PY{k}{def} \PY{n+nf}{\PYZus{}\PYZus{}init\PYZus{}\PYZus{}}\PY{p}{(}\PY{n+nb+bp}{self}\PY{p}{,} \PY{o}{*}\PY{n}{args}\PY{p}{)}\PY{p}{:}
                 \PY{n+nb+bp}{self}\PY{o}{.}\PY{n}{data} \PY{o}{=} \PY{l+s}{\PYZsq{}}\PY{l+s}{\PYZsq{}}\PY{o}{.}\PY{n}{join}\PY{p}{(}\PY{n+nb}{list}\PY{p}{(}\PY{n}{args}\PY{p}{)}\PY{p}{)} 
\end{Verbatim}

    \begin{Verbatim}[commandchars=\\\{\}]
{\color{incolor}In [{\color{incolor}50}]:} \PY{n}{yz} \PY{o}{=} \PY{n}{NotSure}\PY{p}{(}\PY{l+s}{\PYZsq{}}\PY{l+s}{I}\PY{l+s}{\PYZsq{}}\PY{p}{,} \PY{l+s}{\PYZsq{}}\PY{l+s}{Do}\PY{l+s}{\PYZsq{}} \PY{p}{,} \PY{l+s}{\PYZsq{}}\PY{l+s}{Not}\PY{l+s}{\PYZsq{}}\PY{p}{,} \PY{l+s}{\PYZsq{}}\PY{l+s}{Know}\PY{l+s}{\PYZsq{}}\PY{p}{,} \PY{l+s}{\PYZsq{}}\PY{l+s}{What}\PY{l+s}{\PYZsq{}}\PY{p}{,} \PY{l+s}{\PYZsq{}}\PY{l+s}{To}\PY{l+s}{\PYZsq{}}\PY{p}{,}\PY{l+s}{\PYZsq{}}\PY{l+s}{Type}\PY{l+s}{\PYZsq{}}\PY{p}{)}
\end{Verbatim}

    \begin{Verbatim}[commandchars=\\\{\}]
{\color{incolor}In [{\color{incolor}51}]:} \PY{n}{yz}\PY{o}{.}\PY{n}{data}
\end{Verbatim}

            \begin{Verbatim}[commandchars=\\\{\}]
{\color{outcolor}Out[{\color{outcolor}51}]:} 'IDoNotKnowWhatToType'
\end{Verbatim}
        
    \section{Where to go from here?}\label{where-to-go-from-here}

    Practice alone can help you get the hang of python. Give your self
problem statements and solve them. You can also sign up to any
competitive coding platform for problem statements. The more you code
the more you discover and the more you start appreciating the language.

Now that you have been introduced to python, You can try out the
different python libraries in the field of your interest. I highly
recommend you to check out this curated list of Python frameworks,
libraries and software http://awesome-python.com

The official python documentation : https://docs.python.org/2/

Enjoy solving problem statements because life is short, you need python!

Peace.

Rajath Kumar M.P ( rajathkumar dot exe at gmail dot com)


    % Add a bibliography block to the postdoc
    
    
    
    \end{document}
  
    
    \end{document}
  
    
    
    \end{document}
  
    
    \end{document}
